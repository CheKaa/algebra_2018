\section{
 Тензоры. Примеры. Координаты тензора. Замена переменной – случай тензора валентности (1,0). %не писать формулы
}

\textbf{Необработанная версия из конспекта Константина Михайловича}

Дадим определение:

\dfn Тензором валентности $(p,q)$ на пространстве $V$ называется элемент пространства ${V^{*}}^{\otimes p} \otimes V^{\otimes q}$. Так же будем говорить, что такие элементы -- это $p$ раз ковариантные и $q$ раз контравариантные тензоры. Тензорами валентности $(0,0)$ называются элементы поля $K$ -- скаляры.
\edfn

Теперь я утверждаю, что более менее все встречавшиеся нам структуры на векторном пространстве $V$ являются тензорами.



\exm\\
1) Вектор $v\in V$ является 1 раз контравариантным тензором.\\
2) Элемент двойственного пространства $f \in V^{*}$ является 1 раз ковариантным тензором. Вообще ковариантными называют тензоры, которые соответствуют полилинейным формам на пространстве $V$. Это историческая традиция. Точнее:\\
3) Так как пространство ${V^{*}}^{\otimes p} \cong \left(V^{\otimes p}\right)^*\cong \Hom(V,\dots,V,K)$, то тензор валентности $(p,0)$ соответствует полилинейному отображению $V\times\dots \times V \to K$.\\
4) В частности, тензор валентности $(2,0)$ -- это билинейная форма.\\
5) Линейный оператор -- это элемент $\Hom(V,V)\cong V^{*}\otimes V$, то есть тензор валентности $(1,1)$.\\
6) Структура алгебры на $V$ (без требования ассоциативности) задаётся билинейным отображением $V \times V \to V$, то есть линейным отображением $V\otimes V \to V$ или же элементом $V^{*}\otimes V^* \otimes V$, то есть тензором типа $(2,1)$.\\

Как записать тензор в координатах? Выберем базис $e_1,\dots,e_n$ пространства $V$ и возьмём в двойственном пространстве двойственный базис $e^1,\dots,e^n$. Теперь построим базис тензорного произведения ${V^{*}}^{\otimes p}\otimes V^{\otimes q}$. Он имеет вид $e^{j_1}\otimes\dots\otimes e^{j_p}\otimes e_{i_1}\otimes \dots \otimes e_{i_q}$. Тогда произвольный тензор $T$ валентности $(p,q)$ имеет вид 
$$ T= \sum_{\substack{i_1,\dots,i_q \in \ovl{1,n}\\ j_1,\dots,j_p \in \ovl{1,n}} } \,T_{j_1,\dots,j_p}^{i_1,\dots,i_q}\,\, e^{j_1}\otimes\dots\otimes e^{j_p}\otimes e_{i_1}\otimes \dots \otimes e_{i_q}.$$
Элементы $T_{j_1,\dots,j_p}^{i_1,\dots,i_q}$ называются координатами тензора $T$.


Как меняются координаты тензора при замене базиса? Посмотрим сначала на случай тензоров типа $(1,0)$.

\thrm Пусть $e_1,\dots,e_n$ старый базис $V$, а $\hat{e}_1,\dots,\hat{e}_n$ -- новый. Пусть $C$ -- матрица перехода из старого базиса в новый. Тогда матрица перехода из базиса $e^1,\dots,e^n$ в базис $\hat{e}^1,\dots,\hat{e}^n$ есть ${C^{\top}}^{-1}$.
\proof Зафиксируем конвенцию: матрица $C$ это такая матрица, что для всякого вектора $v$ со старыми координатами $x$ и новыми координатами $y$ выполнено, что $y=Cx$. В частности, это выполнено для вектора $e_i$. Это означает, что вектор $e_i=\sum_{j=1}^nC_{ji}\hat{e}_j$. Собственно, это ещё одна характеризация матрицы $C$. Наша задача найти матрицу $D$, со свойством $e^i=\sum D_{ji}\hat{e}^j$. По определению $e^k(e_i)=\delta_{ki}$. Подставим вместо $e^k$ и $e_i$ выражения с матрицами $C$ и $D$. Имеем
$$\delta_{ki}=\sum_{j}C_{ji} \sum_{l}D_{lk}\hat{e}^l(\hat{e}_j)=\sum_{j,l}C_{ji}D_{lk}\cdot\delta_{lj}=\sum_{j}C_{ji}D_{jk}$$
Теперь это равенство можно проинтерпретировать при помощи матричного произведения как 
$$E_n=C^{\top}D,$$
что и доказывает требуемое.
\endproof
\ethrm


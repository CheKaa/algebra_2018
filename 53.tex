\section{
	Мультипликативность и функция Дирихле. Мультипликативность свёртки.
}

Прежде всего заметим, что 
\lm Функция $f$ мультипликативна, тогда и только тогда, когда для $n=p_1^{\alpha_1}\dots p_k^{\alpha_k}$, выполнено $f(n)=\prod f(p_i^{\alpha_i})$.
\elm

\lm Если функция $f\colon \mb N \to \mb C$ мультипликативна и $f(n)=O(n^{\alpha})$, то её ряд Дирихле раскладывается в произведение при $\Re s > \alpha+1$
$$L(s)=\prod_{p \text{ простое}} \left(1+\frac{f(p)}{p^s}+\frac{f(p^2)}{p^{2s}}+\dots\right).$$
\proof Прежде всего заметим, что все ряды $1+\frac{f(p)}{p^s}+\frac{f(p^2)}{p^{2s}}+\dots$ сходятся абсолютно при $\Re s>\alpha+1$. Далее воспользуемся фактом из математического анализа, что произведение двух абсолютно сходящихся рядов сходится к произведению сумм сомножителей. Таким образом, для конечного множества простых $S$ имеет место равенство
$$\prod_{p \in S } \left(1+\frac{f(p)}{p^s}+\frac{f(p^2)}{p^{2s}}+\dots\right)=\sum_{n\in N(S)} \frac{f
(n)}{n^s},$$
где $N(S)$ -- это множество всех натуральных чисел в чьё разложение входят только простые из $S$. Возьмём $S_m=\{p\in \mb N \,|\, p \text{ простое и } p\leq m\}$. Переходя к пределу справа и слева, используя абсолютную сходимость ряда $L(s)$, получаем требуемое.

Обратно, если функция $L(s)$ раскладывается в произведение, то 
$$L(s)=\lim_{m\to \infty} \prod_{p \in S_m } \left(1+\frac{f(p)}{p^s}+\frac{f(p^2)}{p^{2s}}+\dots\right)=\sum_{\substack{ n\in N(S_m) \\ n=p_1^{\alpha_1}\dots p_k^{\alpha_k}}}\!\! \frac{f(p_1^{\alpha_1})\dots f(p_k^{\alpha_k})}{n^s}.$$
Предел правого выражения есть функция Дирихле для $g(n)=f(p_1^{\alpha_1})\dots f(p_k^{\alpha_k})$, если $n=p_1^{\alpha_1}\dots p_k^{\alpha_k}$, причём, как мы знаем, этот ряд Дирихле абсолютно сходится при $s>\alpha +1+\log c$ где $c$ -- константа, что $|f(n)|\leq c n^{\alpha}$, так как 
$$|g(n)|\leq c^k(p_1^{\alpha_1}\dots p_k^{\alpha_k})^{\alpha}=O(n^{\alpha+\log c}).$$
Тогда получаем равенство двух рядов Дирихле в области $s>\alpha+1\log c$. Отсюда $$f(n)=g(n)=f(p_1^{\alpha_1})\dots f(p_k^{\alpha_k}).$$
\endproof
\elm

Если мы перемножим две $L$-функции Дирихле для мультипликативных последовательностей, то получим снова функцию для мультипликативной последовательности. Таким образом должно быть верно

\lm Свёртка двух мультипликативных функций снова мультипликативна.
\proof Для комплексных последовательностей всё и так доказано. Далее можно воспользоваться принципом про продолжимость полиномиальных тождеств. Но несложно доказать мультипликативность свёртки напрямую:
$$f*g(nm)=\sum_{d|nm} f\left(\frac{nm}{d}\right)g(d) = \sum_{d_1 |n, \, d_2|m}f\left(\frac{n}{d_1}\right)f\left(\frac{m}{d_2}\right)g(d_1)g(d_2)= f*g(n)\cdot f*g(m).$$ 
\endproof
\elm



\section{
 Единственность положительного собственного вектора. Применение к случайному блужданию.
}

Шпаргалка: Не забудь, что можем искать предел $\lim\limits_{k\to\infty}A^kv$, если у $ A $ макс по модулю с. ч. $\lambda=1$ кратности 1. $A=P(G)$ нам не походит, замена $P(G)$: $P_{\alpha}(G)=(1-\alpha) P(G) + \alpha\tfrac{1}{n}J,\ \alpha \in (0,1),\ \forall i,j\ J_{ij}=1 $ -- а это норм, Перрон гарантирует.

\lm Пусть $A>0$, $\lambda$ -- максимальное по модулю собственное число. Если у матрицы $A$ есть собственный вектор $y\geq 0$, то $y$ собственный вектор для числа $\lambda$
\elm
\proof Рассмотрим матрицу $A^{\top}$. У неё есть положительный  собственный вектор $x$, соответсвующий собственному числу $\lambda$. Пусть $\mu$ -- собственное число для $y$. Тогда 
$$\lambda x^{\top}y= x^{\top}Ay=x^{\top}\mu y=\mu x^{\top}y.$$
Так как $x^{\top}y >0$, то $\lambda=\mu$.
\endproof

\subsection*{Случайное блуждание}

\textit{(Если я правильно понял идею, то все выглядит так:)}

Пусть у нас есть набор страниц, каждая из которых ссылается на какие-то другие. Также есть пользователь, который произвольно их читает (случайно переходя по ссылкам). Обозначим за $v$ -- его исходное состояние (первую страницу, которую он читает). Хотим для каждой вершины узнать вероятность в ней оказаться после долгого блуждания по графу (вершины -- страницы, ссылки переходы). Это поможет нам отсортировать страницы по "полезности".

\dfn
  Матрица случайного блуждания  $P(G)$:

  $$ P_{ij}=\begin{cases}
  \frac{1}{d_j}, \text{ если есть ребро $j\to i$}\\
  1, \text{ если из вершины не исходит рёбер} \\
  0, \text{ иначе }
  \end{cases}. $$
  $d_j$ -- степень вершины $j$.
\edfn

После одного перехода из $v$ интересующие нас вероятности составляют вектор $P(G)v$. Значит нас интересует $\lim\limits_{n \to \infty}P(G)^nv$. Заметим, что для некоторого типа матриц такой предел известен (см. факт про предел $\rightarrow$ \ref{res20}). Однако чтобы этим воспользоваться, нужно показать, что у $P(G)$ максимальное по модулю собственное число $\lambda=1$, его кратность 1, а все остальные собственные числа $A$ по модулю строго меньше 1. Воспользуемся теоремой Перрона.

Для начала заметим, что матрица $P(G)$ имеет довольно много нулевых компонент. И, строго говоря, теорема Перрона не может быть верна для $P(G)$ всегда. Как же она может помочь? Для этого мы схитрим и немного поменяем задачу. А именно, рассмотрим матрицу $$P_{\alpha}(G)=(1-\alpha) P(G) + \alpha\tfrac{1}{n}J_n,$$
где $J_n$ -- матрица из одних единиц, а $\alpha \in (0,1)$. Тогда матрицы $P_{\alpha}(G)$ являются положительными. С точки зрения блуждающего пользователя это означает, что у него есть два режима -- первый, в котором он находится с вероятностью $1-\alpha$ -- это режим брождения по ссылкам (образующие граф $G$), а второй режим -- переход на случайную страницу. Для матрицы $P_{\alpha}(G)$ выполнены условия теоремы и поэтому она имеет единственное не кратное максимальное собственное число, которое положительно и соответствующий собственный вектор положителен. Покажем, что это собственное число равно 1.

Для этого рассмотрим матрицу $P_{\alpha}(G)^{\top}$. У этой матрицы есть положительный собственный вектор $(1,\dots,1)$ с собственным числом 1. Но тогда это максимальное по модулю собственное число для $P_{\alpha}(G)^{\top}$ и следовательно для $P_{\alpha}(G)$. 

Теперь можем воспользоваться теоремой Перрона и найти предел $P_{\alpha}(G)^kv$, при $k \to \infty$. Он равен $x$ -- некоторому положительному вектору с собственным числом равным 1. Это позволяет приближённо найти $x$. Практически для этого можно взять $k\sim \log n$. Это позволяет заметно сэкономить на вычислениях по сравнению с теоретическим нахождением собственных векторов. Изучая предел $P_{\alpha}(G)$ при $\alpha \to 0$ можно получить информацию и про исходную матрицу.


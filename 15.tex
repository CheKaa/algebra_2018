\section{
 Тензорное произведение. Существование.
}

Шпаргалка: TODO

\dfn Пусть есть набор пространств $V_1, \dots,V_n$. Тогда их тензорным произведением называется пространство 
$V_1\otimes \dots \otimes V_n$ вместе с полилинейным отображением
$$i \colon V_1 \times \dots \times V_n \to V_1 \otimes \dots \otimes V_n,$$
удовлетворяющее условию: для любого полилинейного отображения $h\colon V_1\times \dots \times V_n \to U$ существует единственное линейное отображение 
$$\hat{h}\colon V_1\otimes \dots \otimes V_n \to U,$$
что 
$$\hat{h}\circ i=h.$$
Иными словами, отображение $i$ -- это <<универсальное>> полилинейное отображение.
\edfn 

\thrm Пусть $V_1,\dots,V_n$ -- набор векторных пространств над полем $K$. Тогда имеет место следующая конструкция тензорного произведения:
$$V_1 \otimes \dots \otimes V_n \cong K\lan V_1 \times \dots \times V_n \ran / Rel,$$
где $Rel$ -- это подпространство, порождённое формальными суммами
$$(\dots, \lambda v+u, \dots) - \lambda (\dots,v, \dots) - (\dots, u, \dots).$$ 
\proof

Для удобства обозначим пространство
$$T=K\lan V_1 \times \dots \times V_n \ran / Rel.$$

[ $K\lan V_1 \times \dots \times V_n \ran$ -- пространство, натянутое на множество $V_1 \times \dots \times V_n$, т. е. пространство всех формальных сумм $\sum\lambda_i(v_{1_i}, \dots, v_{k_i})$  ]

Будем обозначать образы элементов $(v_1,\dots, v_n)$ в $T$ как  $v_1\otimes \dots \otimes v_n$. Отображение $$i \colon V_1\times \dots \times V_n \to T$$
отображающее 
$$(v_1,\dots,v_n) \to v_1\otimes \dots \otimes v_n$$
полилинейно по самому определению соотношений из $Rel$. Действительно, для полилинейности нужно, чтобы
$$v_1 \otimes \dots \otimes (\lambda v_i+u_i) \otimes \dots \otimes v_n =
\lambda v_1 \otimes \dots \otimes v_i \otimes \dots \otimes v_n +
v_1 \otimes \dots \otimes u_i \otimes \dots \otimes v_n$$
Это эквивалентно тому, что все формальные суммы из $Rel$ в $T$ должны быть 0.

Пусть теперь дано пространство $U$ и полилинейное отображение $$h \colon V_1\times \dots \times V_n \to U.$$
Построим отображение $\hat{h}$ следующим образом: сначала определим $\hat{\hat{h}}\colon K\lan V_1 \times \dots \times V_n \ran \to U$, а затем покажем, что оно пропускается через $T$. По самому своему определению $K\lan V_1 \times \dots \times V_n \ran$ имеет базисом элементы $(v_1,\dots,v_n)$. Отображение $\hat{\hat{h}}$ достаточно задать на них. Положим $$\hat{\hat{h}}((v_1,\dots,v_n))=h(v_1,\dots,v_n).$$
Покажем, что оно однозначно пропускается через $T$. Единственность очевидна (заданы образы всех базисных). Для того чтобы показать, что $\hat{\hat{h}}$ пропускается через $T$ необходимо показать, что все соотношения лежат в ядре $\hat{\hat{h}}$.
$$\hat{\hat{h}}((\dots, \lambda v_i + u_i, \dots) - \lambda(\dots, v_i, \dots) - (\dots, u_i, \dots))=h(\dots, \lambda v_i + u_i, \dots) - \lambda h(\dots, v_i, \dots) - h(\dots, u_i, \dots) =$$
$$\lambda h(\dots, v_i, \dots) + (\dots, u_i, \dots) - \lambda(\dots, v_i, \dots) - (\dots, u_i, \dots) = 0$$
\endproof
\ethrm

\dfn Будем обозначать элемент $i(v_1,\dots,v_n)=v_1\otimes \dots \otimes v_n$. 
\edfn
\section{
 Тензорное произведение. Существование.
}

\textbf{Необработанная версия из конспекта Константина Михайловича}

В прошлом семестре мы подробно остановились на билинейных операциях и их связи с геометрией, но не разобрали в достаточной степени, что же происходит с общими полилинейными отображениями.

\dfn Пусть есть набор пространств $V_1, \dots,V_n$. Тогда их тензорным произведением называется пространство 
$V_1\otimes \dots \otimes V_n$ вместе с полилинейным отображением
$$i \colon V_1 \times \dots \times V_n \to V_1 \otimes \dots \otimes V_n,$$
удовлетворяющее условию что для любого билинейного отображения из $h\colon V_1\times \dots \times V_n \to U$ существует единственное линейное отображение 
$$\hat{h}\colon V_1\otimes \dots \otimes V_n \to U,$$
что 
$$\hat{h}\circ i=h.$$
Иными словами, отображение $i$ -- это <<универсальное>> полилинейное отображение.
\edfn 

Однако, совершенно непонятно есть такое пространство или нет. Более того, самом его существование нам тоже не сильно поможет. Нам нужна конструкция этого пространства и понимание свойств этой конструкции.

\thrm Пусть $V_1,\dots,V_n$ -- набор векторных пространств. Тогда имеет место следующая конструкция тензорного произведения:
$$V_1 \otimes \dots \otimes V_n \cong K\lan V_1 \times \dots \times V_n \ran / Rel,$$
где $Rel$ -- это подпространство порождённое формальными суммами
$$(\dots, \lambda v_1+v_2, \dots) - \lambda (\dots,v_1, \dots) - (\dots, v_2, \dots).$$ 
\proof Для удобства обозначим пространство
$$T=K\lan V_1 \times \dots \times V_n \ran / Rel.$$ Будем обозначать образы элементов $(v_1,\dots, v_n)$ в $T$ как  $v_1\otimes \dots \otimes v_n$. Отображение $$i \colon V_1\times \dots \times V_n \to T$$
отображающее 
$$(v_1,\dots,v_n) \to v_1\otimes \dots \otimes v_n$$
полилинейно по самому определению соотношений из $Rel$. Пусть теперь дано пространство $U$ и полилинейное отображение $$h \colon V_1\times \dots \times V_n \to U.$$
Построим отображение $\hat{h}$ следующим образом: сначала определим $\hat{\hat{h}}\colon K\lan V_1 \times \dots \times V_n \ran \to U$, а затем покажем, что оно пропускается через $T$. По самому своему определению $K\lan V_1 \times \dots \times V_n \ran$ имеет базисом элементы $(v_1,\dots,v_n)$. Отображение $\hat{\hat{h}}$ достаточно задать на них. Положим $$\hat{\hat{h}}((v_1,\dots,v_n))=h(v_1,\dots,v_n).$$
Покажем, что оно однозначно пропускается через $T$. Как всегда единственность очевидна. Для того чтобы показать, что $\hat{\hat{h}}$ пропускается через $T$ необходимо показать, что все соотношения лежат в ядре $\hat{\hat{h}}$. Но это так, потому что $h$ билинейно! 
\endproof
\ethrm

\dfn Будем обозначать элемент $i(v_1,\dots,v_n)=v_1\otimes \dots \otimes v_n$. 
\edfn


\section{
Сингулярные значения и SVD-разложение.
}

Шпаргалка: TODO

{\bf Напоминание} Евклидово пространство -- векторное пространство $V$ над $\mathbb{R}$ вместе с заданной на нём билинейной симметричной формой $\lan \cdot, \cdot\ran$.

\dfn Пусть $A$ -- линейное отображение $A\colon U \to V$ между евклидовыми пространствами. Тогда сопряжённым отображением к $A$ называется такое линейное отображение $A^{*}\colon V \to U$, что $\lan A^*x,y\ran = \lan x,Ay\ran$ для всех $x\in V$ и $y \in U$.
\edfn

\thrm Сопряжённое линейное отображение единственно. Более того, если в $U$ и $V$ выбрать ортонормированные базисы, то матрица сопряжённого отображения в этих базисах будет равна транспонированной матрице исходного.
\proof Достаточно доказать последнюю часть, чтобы показать единственность и существование. Выберем ортонормированные базисы в $U$ и $V$ -- $u_j$ и $v_i$. Обозначим матрицу $A$ в этом базисе за $X$, а кандидата на $A^*$ за $X^*$.

Для равенства из определения сопряжённости необходимо и достаточно его выполения на базисных векторах (так как скалярное произведение -- билинейная функция). Иными словами необходимо и достаточно, чтобы выполнялось $\lan X^{*}e_i,e_j\ran=\lan e_i,Xe_j\ran$. Но первая часть даёт $X^{*}_{ji}$, а вторая -- $X_{ij}$. Итого необходимо и достаточно, чтобы $X^{*}=X^{\top}$.

$$\lan X^{*}e_i,e_j\ran=(X^*e_i)^{\top}e_j = \pmat X^*_{1i}\\
 \vdots\\
 X^*_{ni} \epmat^{\top}e_j = X^*_{ji}$$
 
 $$\lan e_i,Xe_j\ran=e_i^{\top}Xe_j = e_i^{\top}\pmat X_{1j}\\
 \vdots\\
 X_{mj} \epmat = X_{ij}$$

\endproof
\ethrm


\dfn[Сингулярные значения] Пусть $A$ -- линейное отображение $A\colon U \to V$ между евклидовыми пространствами. Тогда сингулярными значениями $A$ называются числа $\sigma_i=\sqrt{d_i}$, где $d_i>0$ -- положительные собственные числа оператора $A^*A \colon U \to U$. Если же говорить на языке матриц, то для матрицы $X$ её сингулярными значениями будут корни из собственных чисел $X^{\top}X$. 
\edfn

Singular Value Decomposition (сингулярное разложение) поясняет геометрический смысл сингулярных значений.

\thrm[SVD разложение] Пусть $A$ -- линейное отображение $A\colon U \to V$ между евклидовыми пространствами. Тогда существуют такие ортонормированные базисы $U$ и $V$, что матрица $A$ имеет вид 
$$\Sigma = \pmat \sigma_1 &\dots& 0 & 0\\
 \vdots & \ddots &\vdots & \vdots\\
 0 & \dots & \sigma_r & 0\\
 0 &  \dots & 0 & 0 \epmat,$$
 где $r$ -- ранг $A$, числа $\sigma_1, \dots, \sigma_r$ его сингулярные значения.
На языке матриц это означает, что для любой матрицы $X \in M_{m\times n}$ существуют матрицы $L$ -- размера $m$ и $R$ -- размера $n$,  что
$$X= L \Sigma R,$$
 с теми же условиями на $r$ и $\sigma_i$.
 
\proof Рассмотрим оператор $B = A^{*}A$. Существует ортонормированный базис $e_1,\dots,e_n$, в котором оператор $B$ диагонален, с неотрицательными числами на диагонали $d_1\geq\dots\geq d_n\geq 0$. 

[ Почему такой базис существует: Выше доказали, что в некотором ортонормированном базисе оператору соответствует симметричная неотрицательная матрица. Симметричная $\to$ он самосопряженный, значит, есть ортонормированный базис из собственных векторов. Матрица неотрицательная $\to$ все собственные числа тоже. ]

Имеем  $d_i=\sigma_i^2$ для единственного положительного $\sigma_i$. 
Посмотрим на вектора $Ae_i \in U$. Они ортогональны. Действительно
$$\lan Ae_i, Ae_j\ran = \lan A^{*}Ae_i,e_j \ran = \lan d_i e_i,e_j\ran,$$
что равно нулю, если $i\neq j$. В случае $i=j$ получаем $||Ae_i||^2=d_i$. Возьмём 
$$f_i=\frac{Ae_i}{\sqrt{d_i}}$$
и дополним этот набор до ортонормированного базиса пространства $V$. Итого имеем $e_1,\dots,e_n$ ортонормированный базис $U$ и $f_1,\dots,f_m$ -- ортонормированный базис $V$.
Посмотрим на матрицу $A$ в этих базисах. По определению $Ae_i=\sqrt{d_i}f_i$. Это и даёт требуемый вид матрице оператора $A$

Поймем, как выглядит матрица $R$. В нашей конструкции матрица $R$ есть матрица замены из стандартного базиса в базис из собственных векторов $e_i$ матрицы $X^{\top}X$. Если за $C$ обозначить матрицу из столбцов $e_i$ (матрица замены из базиса из собственных векторов $e_i$ в стандартный), то $R=C^{-1}$, но $C$ ортогональна и поэтому можно написать $R=C^{T}$, то есть строки $R$ -- собственные вектора $X^{\top}X$. Часто эти вектора называют правыми сингулярными векторами $X$.

Матрица $L$ -- это матрица замены из базиса $f_i$ в стандартный, то есть матрица из столбцов $f_i$.

\endproof
\ethrm

Наличие такого разложения означает, что для всякого линейного отображения можно так выбрать декартову систему координат, что в этой системе координат это отображение будет выглядеть как растяжение вдоль каких-то осей.
\documentclass[10pt,a4paper,oneside]{book}
\usepackage[a4paper,includeheadfoot,top=10mm,bottom=10mm,left=10mm,right=10mm]{geometry}
\usepackage[utf8]{inputenc}
\usepackage[russian]{babel}
\usepackage{amsmath,amsthm,amssymb,amscd,array}
\usepackage{latexsym}
\usepackage{multicol} % нумерция в нескольких колонках
\usepackage{graphicx} 
%\usepackage{pdfsync}
\usepackage{pgf}
\usepackage{tikz}
\usepackage{tikz-cd}
\usetikzlibrary{arrows,backgrounds,patterns,matrix,shapes,fit,calc,shadows,plotmarks}
\usepackage{hyperref} % гиперссылки
\usepackage{cmap}       % Поддержка поиска русских слов в PDF (pdflatex)
%\usepackage{indentfirst}% Красная строка в первом абзаце
\usepackage{misccorr}
\usepackage{arydshln} % штрихованые линии в массивах
\usepackage{mathtools} % выравнивание в матрицах
\usepackage{ccaption}
\usepackage{fancyhdr}
\usepackage{comment}
\usepackage{xcolor}
\hypersetup{
	colorlinks,
	linkcolor={red!80!black},
	citecolor={blue!50!black},
	urlcolor={red!80!black}
}
% цвета для ссылок


\newtheorem{upr}{Упражнение}
\newtheorem{predl}{Предложение}
\newtheorem{komment}{Комментарий}
\newtheorem{conj}{Гипотеза}
\newtheorem{notation}{Обозначение}


\theoremstyle{definition}
\newtheorem{kit}{Кит}
\newtheorem*{rem}{Замечание}
\newtheorem{zad}{Задача}
\newtheorem*{defn}{{\color{yellow!20!red} Определение}}
\newtheorem*{fact}{Факт}
\newtheorem{thm}{{\color{red!40!black} Теорема}}
\newtheorem*{thmm}{Теорема}
\newtheorem{lem}{Лемма}
\newtheorem{cor}{Следствие}
\newtheorem{utvr}{Утверждение}




\renewcommand{\proofname}{Доказательство}
\renewcommand{\mod}{\,\operatorname{mod}\,}
\renewcommand{\Re}{\operatorname{Re}}
\newcommand{\mf}[1]{\mathfrak{#1}}
\newcommand{\mcal}[1]{\mathcal{#1}}
\newcommand{\mb}[1]{\mathbb{#1}}
\newcommand{\mc}[1]{\mathcal{#1}}
\newcommand{\tbf}[1]{\textbf{#1}}
\newcommand{\ovl}{\overline}
\newcommand{\Spec}{\operatorname{Spec}}
\newcommand{\K}{\operatorname{K_0}}
\newcommand{\witt}{\operatorname{W}}
\newcommand{\gw}{\operatorname{GW}}
\newcommand{\coh}{\operatorname{H}}
\newcommand{\dist}{\operatorname{dist}}
\newcommand{\cl}{\operatorname{Cl}}
\newcommand{\Vol}{\operatorname{Vol}}
\newcommand\tgg{\mathop{\rm tg}\nolimits}
\newcommand\ccup{\mathop{\cup}}
\newcommand{\id}{\operatorname{Id}}
\newcommand{\lcm}{\operatorname{lcm}}
\newcommand{\chr}{\operatorname{char}}
\newcommand{\rk}{\operatorname{rank}}
\DeclareMathOperator{\Coker}{Coker}
\DeclareMathOperator{\Ker}{Ker}
\newcommand{\im}{\operatorname{Im}}
\renewcommand{\Im}{\operatorname{Im}}
\newcommand{\Tr}{\operatorname{Tr}}
\newcommand{\re}{\operatorname{Re}}
\newcommand{\tr}{\operatorname{Tr}}
\newcommand{\ord}{\operatorname{ord}}
\newcommand{\Stab}{\operatorname{Stab}}
\newcommand{\orb}{\operatorname{\mathcal O}}
\newcommand{\Fix}{\operatorname{Fix}}
\newcommand{\Hom}{\operatorname{Hom}}
\newcommand{\End}{\operatorname{End}}
\newcommand{\Aut}{\operatorname{Aut}}
\newcommand{\Inn}{\operatorname{Inn}}
\newcommand{\Out}{\operatorname{Out}}
\newcommand{\GL}{\operatorname{GL}}
\newcommand{\SL}{\operatorname{SL}}
\newcommand{\SO}{\operatorname{SO}}
\renewcommand{\O}{\operatorname{O}}
\renewcommand{\U}{\operatorname{U}}
\newcommand{\Sym}{\operatorname{Sym}}
\newcommand{\Adj}{\operatorname{Adj}}
\newcommand{\Disc}{\operatorname{Disc}}
\newcommand{\cnt}{\operatorname{cont}}
\newcommand{\Frob}{\operatorname{Frob}}

\newcommand{\di}{\mathop{\,\scalebox{0.85}{\raisebox{-1.2pt}[0.5\height]{\vdots}}\,}}

\newcommand{\ndi}{\mathop{\not\scalebox{0.85}{\raisebox{-1.2pt}[0.5\height]{\vdots}}\,}}
\newcommand{\nequiv}{\not \equiv}
\newcommand{\Nod}{\operatorname{\text{НОД}}}
\newcommand{\Nok}{\operatorname{\text{НОК}}}
\newcommand{\sgn}{\operatorname{sgn}}


\def\llq{\textquotedblleft} 
\def\rrq{\textquotedblright} 
\def\exm{\noindent {\bf Примеры:}}


\def\Cb{\ovl{C}}
\def\ffi{\varphi}
\def\pa{\partial}
\def\V{\bf V}
\def\La{\Lambda}
\def\eps{\varepsilon}
\def\del{\delta}
\def\Del{\Delta}
\def\A{\EuScript{A}}
\def\lan{\left\langle }
\def\ran{\right\rangle}
\def\bar{\begin{array}}
	\def\ear{\end{array}}
\def\beq{\begin{equation}}
\def\eeq{\end{equation}}
\def\thrm{\begin{thm}}
	\def\ethrm{\end{thm}}
\def\dfn{\begin{defn}}
	\def\edfn{\end{defn}}
\def\lm{\begin{lem}}
	\def\elm{\end{lem}}
\def\zd{\begin{zad}}
	\def\ezd{\end{zad}}
\def\prdl{\begin{predl}}
	\def\eprdl{\end{predl}}
\def\crl{\begin{cor}}
	\def\ecrl{\end{cor}}
\def\rm{\begin{rem}}
	\def\erm{\end{rem}}
\def\fct{\begin{fact}}
	\def\efct{\end{fact}}
\def\enm{\begin{enumerate}}
	\def\eenm{\end{enumerate}}
\def\pmat{\begin{pmatrix}}
	\def\epmat{\end{pmatrix}}
\def\utv{\begin{utvr}}
	\def\eutv{\end{utvr}}
\def\bupr{\begin{upr}}
	\def\eupr{\end{upr}}
\def\cmm{\begin{comment}}
	\def\ecmm{\end{comment}}

\frenchspacing
\righthyphenmin=2
%\usepackage{floatflt}
\captiondelim{. }




\begin{document}
	\tableofcontents 
	% Некоторые названия билетов обрезаны т.к. они слишком длинные и ломают сборку pdf.
	
	\section{
 Оценка на собственные числа ограничения. Оценка на след.
}

Шпаргалка: TODO

\thmm{Оценка на собственные числа ограничения}

Пусть $q(x)$~--- квадратичная форма на евклидовом пространстве $V$; $q(x) = x^TAx$. $U \le V$. Рассмотрим сужение $q$ на $U$: $q|_U$. Тогда этой форме соответствует оператор $B$ такой, что $q|_U = x^TBx$. Обозначим собственные числа $A$ как $\lambda_i$, при чём $\lambda_i \ge \lambda_{i+1}$. Аналогично собственные числа $B$~--- $\mu_j$; $\mu_j\ge \mu_{j + 1}$. Тогда выполняются неравенства:
$$
\lambda_{i+n-m}\le\mu_i\le\lambda_i
$$

\proofname

$\mu_i\le\lambda_i$

Пусть $dim\,U = m,\ dim\,V = n$. Тогда по теореме Куранта-Фишера:
$$
\lambda_i = \max\limits_{L \le V,\,dim\,L = i} \min\limits_{x\in L} q(x) 
$$

Такое же равенство для $\mu_i$:
$$
\mu_i = \max\limits_{T \le U,\,dim\,T = i} \min\limits_{x\in T} q(x) 
$$

$T\le U\le V\Rightarrow\ T\le V$. Значит, в равенстве для $\lambda_i$ максимум берётся по всем подпространствам, которые учтены в $\mu_i$ и ещё каким-то $\Rightarrow$ $\lambda_i \ge \mu_i$. Что и требовалось.

$\mu_i \ge \lambda_{i + n - m}$

Применим второе равенство из теоремы Куранта-Фишера.

$$
\lambda_{i+n-m} = \min\limits_{L \le V,\,dim\,L = n - (i + n - m) + 1 = m - i + 1} \max\limits_{x\in L} q(x) 
$$

$$
\mu_i = \min\limits_{T \le U,\,dim\,T = m - i + 1} \max\limits_{x\in T} q(x) 
$$

По тем же соображениям $T \le V$. В равенстве для $\lambda_i$ минимум берётся по всем подпространствам, которые учтены в $\mu_i$ и ещё каким-то $\Rightarrow$ $\lambda_{i + n - m}\le \mu_i$. Что и требовалось.

\defn{След квадратичной формы}

Пусть $q$~--- квадратичная форма на евклидовом пространстве $V$. $u_i$~--- ортонормированный базис $V$. Тогда определим след $q$ следующим образом:
$$
Tr\,q=\sum\limits_{i=1}^{dim\,V} q(u_i)
$$

\thmm{Это действительно след!}

Если в базисе $u_i$ форме $q(x)=x^TAx$ соответствует симметричная матрица $A$, то $Tr\,q=Tr\,A$. И след квадратичной формы не зависит от выбора ортонормированного базиса.

\proofname

Посчитаем след матрицы $A$ в базисе $u_i$.

Как вычислить значение $i$-го диагонального элемента матрицы? Возьмём $v_i=(0,\ldots,1,\ldots 0)$ (единица на $v$-ой позиции). Посчитаем $v_iAv_i^T$. Получим $A_{i, i}$. (Упражнение: убедиться, что это действительно так.)

Но $q(u_i) = (0,\ldots,1,\ldots,0)A(0,\ldots,1,\ldots,0)^T = v_iAv_i^T$.

След не зависит от выбора базиса, так как при переходе к новому базису у нас получится $Tr\,CAC^{-1}$, а последнее равно $Tr\,A$, так как $Tr(AB)=Tr(BA)$ (прошлый семестр) и $C$-шки умрут.

\thmm{Оценка на след}

Пусть $U\le V$, где $V$~--- евклидово, $dim\,V = n,\ dim\,U = m$. $q(x)$~--- квадратичная форма на $V$. $q(x)\ge 0$ на всех $x$. Тогда:
$$
Tr(q)\ge Tr(q|_U)
$$

\proofname

След оператора~--- сумма его собственных чисел. По неравенству $\lambda_i\ge\mu_i$ и неотрицательности $q$ получаем $\sum\limits_{i=1}^n \lambda_i \ge \sum\limits_{i=1}^m \mu_i$, а это как раз следы соответствующих операторов (а они равны следам квадратичных форм).

\rem А в будущем потребуется утверждение $\sum\limits_{i = 1}^m \lambda_i \ge \sum\limits_{i = 1}^m \mu_i$ (то есть не для $V$, а для подпространства $V$, натянутого на первые $m$ собственных векторов). Оно верно и без требования $q(x)\ge 0$.


{\bf 6)} Метод главных компонент.

Шпаргалка: TODO

Пусть $V = \mathbb{R}^n$, $x_1,\ldots, x_s \in V$.

Хочется найти афинное подпространство $L\le V$, $dim\,L = k$ такое, что сумма квадратов расстояний от $x_i$ до $L$ (вообще, корень из суммы квадратов, но на него забьём) минимальна.

{\bf Напоминание 1.} $\rho(x, L) = ||pr_{L^{\perp}}(x)||$

{\bf Напоминание 2 (теорема Пифагора).} $||pr_{L^{\perp}}(x)||^2 + ||pr_L(x)||^2 = ||x||^2$

{\bf Напоминание 3.} $e_1,\ldots,e_k$~--- базис $U$ (подпространства $V$). Тогда $pr_U(x)=\sum\limits_{i=1}^k\frac{\langle x, e_i\rangle}{\langle e_i, e_i\rangle}e_i$

---------------------------------------------------------------------------------------------------------------------------------------

{\bf Вывод алгоритма.}

$L = L_0 + a$, где $L_0$~--- линейное. 

Сначала найдём $a$.

\thrm Оптимально взять $a=\frac{1}{s}\sum\limits_{i=1}^s x_i$

\proof

Если из каждого $x_i$ вычесть $a$, то все расстояния нужно будет считать до $L_0$.

$S = \sum\limits_{i = 1}^s (\rho(x_i-a, L_0))^2 = \sum\limits_{i = 1}^s ||pr_{L_0^{\perp}}(x_i-a)||^2$

Возьмём ортонормированный базис $V$ $u_i$ такой, что $u_1,\ldots, u_k$ образуют базис $L_0$, а $u_{k+1},\ldots,u_n$~--- дополнение до базиса.

$L_0^\perp = \langle u_{k+1},\ldots,u_n\rangle$. 

$pr_{L_0^{\perp}}(x_j-a) = \sum\limits_{i = k + 1}^n \frac{\langle x_j - a, u_i\rangle}{\langle u_i, u_i\rangle}u_i = \sum\limits_{i = k + 1}^n \langle x_j - a, u_i\rangle u_i$ (так как $u_i$ ортонормированны, $\langle u_i, u_i\rangle = 1$)

$||pr_{L_0^{\perp}}(x_j-a)||^2 = \langle\sum\limits_{i = k + 1}^n \langle x_j - a, u_i\rangle u_i, \sum\limits_{i = k + 1}^n \langle x_j - a, u_i\rangle u_i\rangle = \sum\limits_{i = k + 1}^n \langle x_j - a, u_i\rangle^2$ (выживают только слагаемые с одинаковыми $u_i$, опять же, из-за ортонормированности)

$\langle x_j - a, u_i\rangle$ равняется $i$-ой координате $x_j - a$. А она, в свою очередь, равна $x_{j, i} - a_i$

$||pr_{L_0^{\perp}}(x_j-a)||^2 = \sum\limits_{i = k + 1}^n (x_{j, i} - a_i)^2$

Мы минимизируем сумму квадратов расстояний по всем $x$. То есть:

$\sum\limits_{j = 1}^s \sum\limits_{i = k + 1}^n (x_{j, i} - a_i)^2$

Продифференцируем по $a_i$ и найдём точку минимума:

$s\cdot 2a_i -2\sum\limits_{j=1}^s x_{j, i} = 0$

$a_i = \frac{1}{s} \sum\limits_{j=1}^s x_{j, i}$

Проделав то же самое по всем $i$, получим $a = \frac{1}{s} \sum\limits_{j=1}^s x_j$. Это и хотели.

\ethrm

---------------------------------------------------------------------------------------------------------------------------------------

{\bf Поиск подходящего линейного подпространства $L_0$}.

Мы хотим минимизировать сумму квадратов расстояний. Это (по теореме Пифагора (напоминание 2)) то же самое, что максимизировать сумму квадратов проекций на $L$ (так как квадрат нормы $x$ у нас фиксирован).

$u_i$~--- ортонормированный базис $L_0$.

$S = \sum\limits_{i = 1}^s ||pr_{L_0}(x_i)||^2$

$pr_L(x_j) = \sum\limits_{i = 1}^k \langle x_j, u_i\rangle u_i$ (по тем же соображениям, что и в прошлом пункте).

$S = \sum\limits_{i = 1}^s\langle\sum\limits_{i = 1}^k \langle x_j, u_i\rangle u_i, \sum\limits_{i = 1}^k \langle x_j, u_i\rangle u_i\rangle = \sum\limits_{i = 1}^s\sum\limits_{j = 1}^k \langle x_i, u_j\rangle^2 = \sum\limits_{j = 1}^k\sum\limits_{i = 1}^s \langle x_i, u_j\rangle^2$ ($u$-шки ушли, так как базис ортонормированный).

Определим матрицу $X$ следующим образом:
$$
X = \pmat
x_1^T\\
x_2^T\\
\vdots\\
x_s^T
\epmat
$$

Это матрица размера $s\times n$

Посмотрим на результат $X\cdot u_j$:
$$
A=\pmat
\langle x_1, u_j\rangle\\
\vdots\\
\langle x_s, u_j\rangle
\epmat 
$$

Заметим, что если мы запишем $u_j^T\cdot X^T$, то получим $A^T$.

$A^T\cdot A = u_j^TX^TXu_j = \sum\limits_{i = 1}^s\langle x_i, u_j\rangle^2$.

То есть, это значение квадратичной формы с матрицей $X^TX$ на векторе $u_j$.

$q(u) = u^TX^TXu$

Теперь $S = \sum\limits_{i = 1}^k q(u_i) = Tr\, q(x)|_{L_0}$

По замечанию из неравенства на след (см. предыдущий билет) $S = Tr\,q(x)|_{L_0} \le \sum\limits_{i = 1}^k \lambda_i$. Но мы максимизируем $S$ и знаем, на каком подпространстве достигается $\sum\limits_{i = 1}^k \lambda_i$ (это подпространство $\langle v_1,\ldots, v_k\rangle$, где $v_i$~--- собственный вектор, соотвествующий собственному числу $\lambda_i$).

Таким образом, $L_0 = \langle v_1,\ldots,v_k\rangle$, где $v_i$~--- собственный вектор, соответствующий $i$-ому по убыванию собственному числу матрицы $X^TX$.

---------------------------------------------------------------------------------------------------------------------------------------

{\bf Алгоритм (обобщение).}

Ищем афинное подпространство $L = L_0 + a$.

$L_0$~--- линейное, $dim\,L_0 = k$

$a = \frac{1}{s}\sum\limits_{i = 0}^s x_i$

Из каждого $x_i$ вычли $a$.

Для новых векторов $x$ построили матрицу 
$$
X=\pmat
x_1^T\\
x_2^T\\
\vdots\\
x_s^T
\epmat
$$

Нашли матрицу $C = X^TX$. Нашли её собственные числа $\lambda_1\ge\lambda_2\ge\ldots\ge\lambda_n$ и собственные вектора $v_1,\ldots, v_n$, им соответствующие.

$L_0 = \langle v_1,\ldots,v_k\rangle$

---------------------------------------------------------------------------------------------------------------------------------------

7) Сингулярные значения и SVD-разложение.


8) Приближение матрицей указанного ранга и SVD-разложение. Возможность применения к сжатию изображения.


9) Положительные матрицы. Теорема Перрона.


\section{
 Единственность положительного собственного вектора. Применение к случайному блужданию.
}

Шпаргалка: Не забудь, что можем искать предел $\lim\limits_{k\to\infty}A^kv$, если у $ A $ макс по модулю с. ч. $\lambda=1$ кратности 1. $A=P(G)$ нам не походит, замена $P(G)$: $P_{\alpha}(G)=(1-\alpha) P(G) + \alpha\tfrac{1}{n}J,\ \alpha \in (0,1),\ \forall i,j\ J_{ij}=1 $ -- а это норм, Перрон гарантирует.

\lm Пусть $A>0$, $\lambda$ -- максимальное по модулю собственное число. Если у матрицы $A$ есть собственный вектор $y\geq 0$, то $y$ собственный вектор для числа $\lambda$
\elm
\proof Рассмотрим матрицу $A^{\top}$. У неё есть положительный  собственный вектор $x$, соответсвующий собственному числу $\lambda$. Пусть $\mu$ -- собственное число для $y$. Тогда 
$$\lambda x^{\top}y= x^{\top}Ay=x^{\top}\mu y=\mu x^{\top}y.$$
Так как $x^{\top}y >0$, то $\lambda=\mu$.
\endproof

\subsection*{Случайное блуждание}

\textit{(Если я правильно понял идею, то все выглядит так:)}

Пусть у нас есть набор страниц, каждая из которых ссылается на какие-то другие. Также есть пользователь, который произвольно их читает (случайно переходя по ссылкам). Обозначим за $v$ -- его исходное состояние (первую страницу, которую он читает). Хотим для каждой вершины узнать вероятность в ней оказаться после долгого блуждания по графу (вершины -- страницы, ссылки переходы). Это поможет нам отсортировать страницы по "полезности".

\dfn
  Матрица случайного блуждания  $P(G)$:

  $$ P_{ij}=\begin{cases}
  \frac{1}{d_j}, \text{ если есть ребро $j\to i$}\\
  1, \text{ если из вершины не исходит рёбер} \\
  0, \text{ иначе }
  \end{cases}. $$
  $d_j$ -- степень вершины $j$.
\edfn

После одного перехода из $v$ интересующие нас вероятности составляют вектор $P(G)v$. Значит нас интересует $\lim\limits_{n \to \infty}P(G)^nv$. Заметим, что для некоторого типа матриц такой предел известен (см. факт про предел $\rightarrow$ \ref{res20}). Однако чтобы этим воспользоваться, нужно показать, что у $P(G)$ максимальное по модулю собственное число $\lambda=1$, его кратность 1, а все остальные собственные числа $A$ по модулю строго меньше 1. Воспользуемся теоремой Перрона.

Для начала заметим, что матрица $P(G)$ имеет довольно много нулевых компонент. И, строго говоря, теорема Перрона не может быть верна для $P(G)$ всегда. Как же она может помочь? Для этого мы схитрим и немного поменяем задачу. А именно, рассмотрим матрицу $$P_{\alpha}(G)=(1-\alpha) P(G) + \alpha\tfrac{1}{n}J_n,$$
где $J_n$ -- матрица из одних единиц, а $\alpha \in (0,1)$. Тогда матрицы $P_{\alpha}(G)$ являются положительными. С точки зрения блуждающего пользователя это означает, что у него есть два режима -- первый, в котором он находится с вероятностью $1-\alpha$ -- это режим брождения по ссылкам (образующие граф $G$), а второй режим -- переход на случайную страницу. Для матрицы $P_{\alpha}(G)$ выполнены условия теоремы и поэтому она имеет единственное не кратное максимальное собственное число, которое положительно и соответствующий собственный вектор положителен. Покажем, что это собственное число равно 1.

Для этого рассмотрим матрицу $P_{\alpha}(G)^{\top}$. У этой матрицы есть положительный собственный вектор $(1,\dots,1)$ с собственным числом 1. Но тогда это максимальное по модулю собственное число для $P_{\alpha}(G)^{\top}$ и следовательно для $P_{\alpha}(G)$. 

Теперь можем воспользоваться теоремой Перрона и найти предел $P_{\alpha}(G)^kv$, при $k \to \infty$. Он равен $x$ -- некоторому положительному вектору с собственным числом равным 1. Это позволяет приближённо найти $x$. Практически для этого можно взять $k\sim \log n$. Это позволяет заметно сэкономить на вычислениях по сравнению с теоретическим нахождением собственных векторов. Изучая предел $P_{\alpha}(G)$ при $\alpha \to 0$ можно получить информацию и про исходную матрицу.



12) Сильно регулярные графы. Граф Петерсона и его спектр. Двудольность и спектр.\\
Шпаргалка: TODO\\


13) Две оценки на размер максимального независимого множества.\\    
Натянуть подпространство на  множество, следствие из КФ, нулевая квадратичная форма\\
Характеристический вектор множества, разложить по ортонорм. базису регулярного(!) графа с $u_1 = (1,\cdots,1) \frac{1}{\sqrt n}$\\


14) $K_{10}$ не покрывается тремя Петерсонами.\\
$\sum\limits_{i=1}^3 A_i = B$. Все рег $\Rightarrow$ общий с.в. $(1,\dots,1)$ для $P$ с.ч. 3, для полного с.ч. $9$. Сузим. Для $A_1$ и $A_2$ подпр. порожд. с.в. с с.ч. 1 $\cap$. Распишем для $u$ из $\cap$. $Bu = -u$ (натянуто на с.в. с с.ч. $-1$). $\Rightarrow$ с.в. для $A_3$ с с.ч. $-3$. Такого с.ч. нет.\\


15) Тензорное произведение. Существование.


16) Единственность тензорного произведения. Размерность тензорного произведения.\\

Шпаргалка: TODO\\


17) Тензорное произведение линейных отображений. Кронекерово произведение. Тензорное произведение операторов и его собственные числа. Категорное произведение графов.\\

Единств: определено на тензорятах; $\exists:$ отобразить $U_1\times\dots\times U_k$ в $V_1\otimes\dots\otimes V_K$ полилин. (композ полилин.) $\Rightarrow$ (опр. тенз.) $\exists!$. Наше правило подходит. Матрица: расписать $(\sum\limits_{k} A_{k, i} f_k)\otimes(\sum\limits_{l} B_{l, j} f_l')$. С.ч. $A\otimes B$: жорданов базис.\\


18) Канонические изоморфизмы для тензорного произведения.


\section{
 Тензоры. Примеры. Координаты тензора. Замена переменной – случай тензора валентности (1,0).
}


\section{
 Замена переменной – общий случай.
}

\textbf{Необработанная версия из конспекта Константина Михайловича}

Теперь мы можем разобраться с тензорами общего вида:

\thrm Пусть $e_1,\dots,e_n$ старый базис $V$, а $\hat{e}_1,\dots,\hat{e}_n$ -- новый. Пусть $C$ -- матрица перехода из старого базиса в новый, а $D={C^{\top}}^{-1}$. Тогда координаты тензора $T$ в базисе $\hat{e}$ выражаются через старые координаты следующим образом:
$$\hat{T}_{j_1,\dots,j_p}^{i_1,\dots,i_q}=\sum_{\substack{i'_1,\dots,i'_q \in \ovl{1,n}\\ j'_1,\dots,j'_p \in \ovl{1,n}}} \,\,
\prod_{t\in \ovl{1,p}} D_{j_t,j'_t} \prod_{s\in \ovl{1,q}} C_{i_s,i'_s}  \,\,T_{j'_1,\dots,j'_p}^{i'_1,\dots,i'_q}.$$
\proof Обозначим за $e^{j_1,\dots,j_p}_{i_1,\dots,i_q}$ тензор $e^{j_1}\otimes \dots \otimes e^{j_p} \otimes e_{i_1}\otimes \dots \otimes e_{i_q}$. Рассмотрим тензор
$$ T= \sum_{\substack{i'_1,\dots,i'_q \in \ovl{1,n}\\ j'_1,\dots,j'_p \in \ovl{1,n}}} T_{j'_1,\dots,j'_p}^{i'_1,\dots,i'_q} e^{j'_1,\dots,j'_p}_{i'_1,\dots,i'_q}$$ 
и заменим $e_i=\sum_{j=1}^nC_{ji}\hat{e}_j$ и $e^i=\sum D_{ji}\hat{e}^j$. Получится такая сумма:
$$ T= \sum_{\substack{i'_1,\dots,i'_q \in \ovl{1,n}\\ j'_1,\dots,j'_p \in \ovl{1,n}}} T_{j'_1,\dots,j'_p}^{i'_1,\dots,i'_q} \sum_{\substack{i_1,\dots,i_q \in \ovl{1,n}\\ j_1,\dots,j_p \in \ovl{1,n}} } D_{j_1j'_1}\dots D_{j_pj'_p} C_{i_1i'_1}\dots C_{i_qi'_q} \hat{e}^{j_1,\dots,j_p}_{i_1,\dots,i_q}$$
Осталось поменять суммирование местами.
\endproof
\ethrm

Важность тензоров в теоретической физике обуславливается тем, что практически все физические объекты -- это тензоры. Точнее: с точки зрения теории относительности пространство-время это некоторое четырёхмерное многообразие $M$ (в двумерной ситуации подошла бы обычная сфера или тор). С каждой точкой $x$ этого многообразия связано касательное пространство в этой точке -- некоторое четырёхмерное пространство $T_x$. Представим себе, что в каждой точке пространства задана плотность вещества (на самом деле не так, но допустим) -- это даёт вам функцию $f \colon M \to \mb R$ -- скаляр в каждой точке, то есть тензор типа $(0,0)$. 

Направление движения материи можно задать взяв в каждой точке касательный вектор, то есть тензор ранга $(0,1)$ на $T_x$. Дальше, у каждого такого вектора можно считать его <<длину>> и углы между векторами. Для этого надо задать для каждой точки $x$ билинейную форму на касательном пространстве, то есть элемент $T_x^{(2,0)}$. И т.д. Чаще всего такие объекты называют тензорными полями, если хочется подчеркнуть, что в разных точках это тензор вообще говоря на разных пространствах.

Важно, что уравнения в физике не должны зависеть от выбора координат. Можно, конечно, писать какие-то уравнения при помощи координат тензоров и каждый раз проверять, что выбрав новые координаты уравнение будет того же вида. Однако, чем сложнее наука тем сложнее становятся проверки. Становится важно работать с тензорами не рассматривая их координаты. Для этого мы обсудим две операции с тензорами, которые легко можно понять не используя координаты. Начнём с самой простой -- умножение тензоров.



21) Тензорная алгебра. Свёртка и след.\\
Для $(1, 1)$ $T=\sum_{i,j} T_j^i e^j\otimes e_i$. Тогда $Conv(T)=\sum_{i,j} T_j^i e^j(e_i)=\sum_i T^i_i.$. $V^*\otimes V \cong \Hom(V,V)\Rightarrow$ это след.


\section{
 \texorpdfstring{Лемма Гаусса. Содержание многочлена. Делимость в $Q(R)[x]$ и в $R[x]$.} %не надо убирать \texorpdfstring
}


\section{
 Факториальность кольца многочленов над факториальным кольцом.
}

\textbf{Необработанная версия из конспекта Константина Михайловича}


\thrm Пусть $R$ -- факториальное кольцо. Тогда кольцо $R[x]$ факториально. Более того, имеет место следующее описание простых элементов кольца $R[x]$:\\
1)  $\cnt(f)=1$ и $f$ неприводим в $Q(R)[x]$.\\
2) $f=p \in R$ -- простой в $R$.
\proof 
Для начала покажем, что все указанные ситуации приводят к простым элементам в кольце $R[x]$ и что других простых и, более того, неприводимых не бывает.
Итак, пусть $f \in R[x]$ неприводим в $Q(R)[x]$. Если $gh\di f$, то это же верно над $Q(R)$ и, можно считать например, что $g\di f$ в $Q(R)[x]$. Тогда $g= fk$. Теперь можно домножить на подходящую константу $g= (cf) (c^{-1}k)$ чтобы получить равенство в $R[x]$. Заметим, что $c(c^{-1}k)$ из $R[x]$, что показывает, что $g \di f$ в $R[x]$. Второй случай полностью следует из леммы Гаусса.


Теперь покажем, что любой элемент раскладывается в произведение указанных простых. Для этого сначала разложим $f$ в $Q(R)[x]$ в произведение неприводимых $f=\prod g_i$, $g_i \in Q(R)[x]$. Далее сделаем из $g_i$ элементы $\hat{g}_i$ из $R[x]$ с $cont(g_i)=1$, что $f=a\prod \hat{g}_i$. Заметим, что $a=cont(f)$ и, следовательно, лежит в $R$. Итого $f=a\prod \hat{g}_i$, где $ 0 \neq c \in R$. Осталось разложить $c$.

Осталось показать единственность. Это следует лишь из того, что у нас есть разложение на простые. Действительно, если $f=\prod p_i=\prod q_i$, то $p_i \di \prod q_i$ и благодаря простоте делит скажем $q_i$. Но тогда $p_ih=q_i$, откуда, благодаря неприводимости $q_i$ получаем, что $h$ обратим, то есть, что $p_i \sim q_i$. Тогда можно сократить на $p_i$ и продолжить по индукции. Отсутствие простых отличного от указанных типов следует теперь из единственности разложения.
\endproof
\ethrm



29) Редукционный признак неприводимости. Примеры. Признак Эйзенштейна.\\
1. $a_n \ndi p$, f - неприводим в $R/p[x]$ $\Rightarrow$ неприводим над $Q(R)$. $cont = 1$ и непр-ть над $Q(R)$ $\Rightarrow$ непр-ть над $R$ (см. степени $g$ и $h$). 2. $a_n \ndi p$, все $a_i \di p$ $i<n$, но $a_0\ndi p^2$, то многочлен $f(x)$ неприводим. Пусть $b_0 \ndi p$, см. min $c_s \ndi p$ и $a_s$.\\



\section{
 Алгоритм Кронекера. Сведение для многочленов от нескольких переменных.
}
{\it Шпаргалка.}\\
 1) Перебираем наборы делителей $f(i)$, $0 \le i \le \frac{deg f}{2}$, интерполируем, проверяем. 2) Различным разложениям $f(x_1,\dots,x_n)$ соответствуют различные разложения $f(x, \dots, x^{d^{n-1}})$ для $d$ больших $\max_{i=1}^n \{\deg_{x_i} f\}$. Рассмотреть образ $x^\alpha$.\\

{\bf Алгорим Кронекера}\\
Итак, пусть есть целочисленный многочлен $f(x)$ и мы хотим разложить его на множители. Мы будем искать разложение на целочисленные многочлены, заметим, что хотя бы один из них имеет степень меньшую, чем $[\frac{n}{2}]$. Вспомним о задаче интерполяции. Если $g$ -- искомый делитель $f$, то $g$ определяется своими значениями в $[\frac{n}{2}]+1$ точке, например в точках $0,1,\dots, [\frac{n}{2}]$. Более того, $f(i) \di g(i)$. Таким образом набор $g(0),\dots, g([\frac{n}{2}])$ состоит из делителей $f(0),\dots,f([\frac{n}{2}])$. Найти все такие наборы -- конечный перебор. По каждому набору восстановим $g$ по интерполяции и проверим, является ли он 
делителем $f$.

Сведем задачу разложения многочленов от нескольких переменных к предыдущей.

\thrm Пусть $R$ -- кольцо. Тогда различным разложениям $f(x_1,\dots,x_n)\in R[x_1,\dots,x_n]$   соответствуют различные разложения $\hat{f}=f(x, x^d, x^{d^2}, \dots, x^{d^{n-1}})$ для $d$ больших $\max_{i=1}^n \{\deg_{x_i} f\}$.
\proof Пусть $f=g_1h_1=g_2h_2$ и пусть $g_1\neq g_2$. Покажем, что $\hat{g_1}\neq \hat{g_2}$. Мы рассматриваем отображение $f(x_1, \dots, x_n) \rightarrow 
f(x, x^d,\dots, x^{d^{n-1}})$. Рассмотрим мономом $x^{\alpha}$, где $\alpha$ - это мультииндекс. Он переходит в многочлен $x^{\alpha_1+\alpha_2d+\dots+\alpha_n d^{n-1}}$. По условию все $\alpha_i<d$ как степени при переменных $x_i$. Тогда моном $x^{\alpha_1+\alpha_2d+\dots+\alpha_n d^{n-1}}$ может быть получен только из монома $x^{\alpha}$. Заметим теперь, что $\deg_{x_i} g_j \leq \deg f <d$. Следовательно мономы многочленов $g_j(x)$ так же однозначно восстанавливаются по мономам $\hat{g_j}$.
\endproof
\ethrm

К сожалению, не стоит ожидать взаимооднозначного соответствия между разложениями многочленов $f$ и $\hat{f}$. Например, многочлен $x_2^2$ раскладывается на два множителя одним способом. При $d=3$ его образ есть $x^6$ у которого 3 различных разложения.

\section{
 Лемма Гензеля. Разложение на множители при помощи леммы Гензеля.
}

\textbf{Необработанная версия из конспекта Константина Михайловича}


К сожалению, не стоит ожидать взаимооднозначного соответствия между разложениями многочленов $f$ и $\hat{f}$. Например, многочлен $x_2^2$ раскладывается на два множителя одним способом. При $d=3$ его образ есть $x^6$ у которого 3 различных разложений.

Теперь вернёмся к многочленам от одной переменной. Для проверки неприводимости мы с успехом использовали информацию, полученную из разложения по модулю $n$. Вопрос -- нельзя ли её же использовать и в целочисленной задаче? 

Во-первых, если взять достаточно большой модуль $n$, заметно больший, чем коэффициенты в целочисленном разложении, то разложение $f$ по модулю $n$ с маленькими коэффициентами однозначно будет определять кандидата на целочисленное разложение. Это соображение встречается сразу с двумя проблемами -- первая -- не ясно какие есть ограничения на коэффициенты сомножителей, вторая -- разложений по модулю $n$ может быть много и нет способа эффективно искать их.



Как же теперь выбрать достаточно большое число, по модулю которого раскладывать многочлен $f$ на множители? В первую очередь, должно быть удобно раскладывать многочлен по подходящему множителю. Наибольшим удобством в решении задачи разложения обладают поля. В этом смысле возможно стоило бы искать разложение $f$ по модулю очень большого простого. Однако найти большое простое число довольно тяжело. Смотреть по модулю маленьких простых а потом пытаться склеивать разложение в духе китайской теоремы об остатках может банально не получиться (как в примере 3 -- неясно во что склеить два разных разложения). Оказывается наиболее оптимальный вариант такой -- взять небольшое простое число $p$, разложить $f$ над $\mb Z/p$ а затем <<поднять>> это разложение по модулю $p^k$ для достаточно большого $k$. Сформулируем утверждение, которое пояснит как это сделать.

\lm[Гензеля] Пусть $f \in \mb Z[x]$, со старшим коэффициентом не делящимся на простое число $p$. Пусть $\ovl{f}=gh$ в кольце $\mb Z/p[x]$, причём $(g,h)=1$. Тогда  для любого $k\geq 1$ существуют единственные по модулю $p^k$ многочлены $\hat{g}, \hat{h} \in \mb Z[x]$, что $\ovl{f}=\hat{g} \hat{h} \mod p^k$  и $\deg h= \deg \hat{h}$, $\deg g= \deg \hat{g}$, $\hat{g}\equiv g \pmod{p}$, а $\hat{h}\equiv h \pmod{p}$.
\proof Докажем это индукцией по $k$. Пусть по модулю $p^{k}$ уже построены подходящие многочлены $\hat{h}$ и $\hat{g}$ и мы хотим построить $\ovl{h}$ и $\ovl{g}$. Заметим, что благодаря единственности по модулю $p^k$, такие $\ovl{g}$ и $\ovl{h}$ обязаны совпадать с $\hat{h}$ и $\hat{g}$ по модулю $p^k$. Это означает, что по модулю $p^{k+1}$ 
$$\ovl{h} \equiv \hat{h}+p^ka(x)\pmod{p^{k+1}} \text{\,\, и\quad } \ovl{g} \equiv \hat{g} + p^kb(x)\pmod{p^{k+1}}.$$
Заметим, что многочлены $a(x)$ и $b(x)$  однозначно определяются по модулю $p$,  если известны $\ovl{g}$ и $\ovl{h}$ и по модулю $p$ могут иметь степени меньше чем степени $h(x)$ и $g(x)$ соответственно. Покажем, что такие $a(x), b(x)$ существуют и единственны по модулю $p$. Заметим, что необходимо проверить лишь условие $f \equiv \ovl{g}\ovl{h} \pmod{p^{k+1}}$. Распишем
$$f\equiv \hat{g}\hat{h} + p^{k}(a(x)g + b(x)h) \pmod{p^{k+1}}.$$
Здесь мы заменили $\hat{h}$ и $\hat{g}$ по модулю $p$ и получили исходные многочлены $g$ и $h$ из $\mb Z/p[x]$. Заметим, что есть единственный такой многочлен $c(x)\in\mb Z/p[x]$, что $f-\hat{g}\hat{h}=p^kc(x) \pmod{p^{k+1}}$. Теперь для выполнения сравнения выше необходимо, чтобы  $$c(x)=a(x)g(x)+b(x)h(x)$$
У такого сравнения есть единственное решение в $\mb Z/p[x]$ при условии $\deg a(x)<\deg h(x)$ и $\deg b(x)< \deg g(x)$. Что и требовалось.
\endproof
\elm


Частным случаем разложения на множители служит разложение вида $f(x)=(x-x_1)g(x)$, соответствующее наличию корня. Сформулируем следствие леммы Гензеля в этой ситуации:


\crl Пусть $f \in \mb Z[x]$, со старшим коэффициентом не делящимся на простое число $p$. Пусть $a$ корень $f$ по модулю $p$, причём $\ovl{f}'(a)\neq 0$. Тогда  для любого $k\geq 1$ существует единственный $ \hat{a}\in \mb Z/p^k$,  что $f(\hat{a})=0$ и $\hat{a} \equiv a \mod p$.
\proof
\endproof
\ecrl

\rm Можно усилить лемму Гензеля, рассматривая подъём разложения не с модуля $p$, а с модуля $p^k$ заработав ослабление условия на производную.
\erm


Теперь алгоритм разложения на множители уже вырисовывается. Но в лемме Гензеля есть некоторые ограничения на разложение многочлена $\ovl{f}$. Как с этим жить мы узнаем дальше.



32) Степенные суммы. Тождество Ньютона.


33) Целые алгебраические элементы. Замкнутость относительно операций.\\
а алгебраический == $\exists f \in \mb Z[x] : f(a) = 0$. Замкнуто: $\prod (x - (a_i + b_j))$ симметрично по $а_i$, тогда коэффициенты выражаются через симметрические, симметрический по $b_i$ - все коэффициенты целые.\\


\section{
 Описание наименьшего подрасширения, содержащего данный элемент.


$K(\alpha) \cong K[\alpha] \cong K[x]/p(\alpha)$, рассмотрим $K[x] \to L$, переводящий $x \to \alpha$ и $K[x]/p(x) \to L$.
Следствия про равенство степеней расширения над $K$ и изоморфность расширений для корней неприводимого многочлена.

\dfn Элемент $\alpha \in L$ называется алгебраическим над $K$, если существует многочлен $0\neq p(x)\in K[x]$, что $p(\alpha)=0$. 
\edfn

\thrm Пусть $L/K$ расширение полей, а $\alpha \in L$. Тогда если $\alpha$ алгебраическое над $K$, то $$K(\alpha)=K[\alpha]\cong K[x]/p(x),  \text{ где $p(x)$ минимальный многочлен для $\alpha$}.$$
Если же $\alpha$ не алгебраическое, то $$K[\alpha]\cong K[x] \text{ и } K(\alpha) \cong K(x).$$
($K(x)$ -- поле дробно-рациональных функций)
\ethrm

\proof Итак, пусть $\alpha$ -- алгебраический над $\mb K$. Тогда минимальный многочлен $\alpha$ однозначно определён. Покажем, что он неприводим. Пусть $p(x)=h(x)q(x)$. Тогда $h(\alpha)q(\alpha)=0$. Но $L$ -- поле. Откуда либо $h(\alpha)=0$ либо $q(\alpha)=0$. Но тогда $p(x)$ не минимальный. Теперь мы знаем, что $K[x]/p(x)$ -- поле.

Существует единственный гомоморфизм $K[x]\to L$, переводящий $x \to \alpha$ и оставляющих $K$ на месте, это гомоморфизм алгебр, его ядро -- $<p(x)>$. Можем определить гомоморфизм $\ffi: K[x]/p(x) \to L$, $\Im \ffi \cong K[x]/p(x)$. 
Понятно, что такой гомоморфизм может быть только единственным.

$\im \ffi$ это алгебра и состоит из линейных комбинаций $1,\alpha,\dots,\alpha^{n-1}$, где $n$ -- степень $p(x)$, так как он изоморфен $K[x]/p(x)$. Тогда $K[\alpha] ~=~ <1, \alpha, \alpha^2, \ldots> ~=~ <1, \alpha, \ldots, \alpha^{n-1}> ~=~ \im \ffi$. $\im \ffi$ -- поле, содержащее $\alpha$ $\Rightarrow$ $\im \ffi \supset K(\alpha) \supset K[\alpha]$, но мы уже поняли, что $K[\alpha] = \im \ffi$. Отсюда $$K[x]/p(x) \cong \im \ffi = K[\alpha]=K(\alpha).$$

Пусть $\alpha$ не алгебраический, то есть трансцендентный. Тогда отображение $\ffi : K[x] \to L$ переводящее $x\to\alpha$ инъективно, так как $\Ker \ffi = \{0\}$. $\im \ffi = <1, \alpha, \ldots>$ -- это $K[\alpha]$.\\ Далее заметим, что cущствует единственное отображение $K(x) \to L$, $\frac{f(x)}{g(x)} \to \frac{f(\alpha)}{g(\alpha)}$, потому что образы всех элементов $K[x]$, кроме 0 в $L$ обратимы. Образ этого отображения есть подполе изоморфное $K(x)$ и имеет вид $\{\frac{f(\alpha)}{g(\alpha)}\}$. Любое поле, которое содержит $\alpha$ должно содержать все такие элементы. Понятно, что это и есть $K(\alpha)$.   
\endproof

\crl Пусть $\alpha$ -- алгебраическое. Тогда $\deg K[\alpha]= \deg K[x]/p(x)= \deg p(x)$, где $p(x)$ -- минимальный многочлен $\alpha$.
\ecrl

\crl Все расширения, порождённые над $K$ корнем одного и того же неприводимого многочлена изоморфны. Часто я буду говорить, про расширение $K[\alpha]$, где $\alpha$ корень многочлена $p(x)$. Это корректно, так как такое расширение определено однозначно с точностью до изоморфизма.
\ecrl
}


\section{
 Построение при помощи циркуля и линейки. Пример неразрешимого построения.
}

38) Построение при помощи циркуля и линейки. Пример неразрешимого построения.


39) Конечные поля. Число элементов. Основное уравнение. Эндоморфизм Фробениуса. Корни $x^{p^n} - x$ образуют подполе.


40) Основная теорема про конечные поля.


41) Подполя данного конечного поля. Описание автоморфизмов $F_p^n$.


\section{
 Расширения поля, неприводимые многочлены... %не надо писать сюда формулы и ldots
}

\textbf{Расширения поля $F_q$ . Неприводимые многочлены как делители $x^{q^d} - x$.}

\textbf{Необработанная версия из конспекта Константина Михайловича}

\thrm Все расширения поля $\mb F_q$, где $q=p^n$ имеют $q^m$ элементов. Два расширения $\mb F_q$ из $q^m$ элементов изоморфны между собой. Внутри поля $\mb F_{q^m}$ есть подполе $\mb F_{q^l}$ только если $l|m$.
\proof
Если $L$ расширение $\mb F_q$, то оно имеет $q^{[L:\mb F_q]}$ элементов. Покажем существование. Возьмём поле из $q^m=p^{nm}$ элементов и рассмотрим в нём подполе $\mb F_q$ из $q=p^n$ элементов. Такое есть по предыдущей теореме. Это и даёт необходимое расширение. 

Покажем единственность такого расширения с точностью до $\mb F_q$ изоморфизма. Основная сложность состоит в том, чтобы проследить за сохранением $\mb F_q$ коэффициентов. Итак, пусть $L_1$ и $L_2$ -- расширения $\mb F_q$ из $q^m$ элементов. Такие поля изоморфны над $\mb F_p$. Пусть $\ffi \colon L_1 \to L_2$ изоморфизм над $\mb F_p$. Вообще говоря он не обязан переводить элементы из $\mb F_q$ в себя же. Нам надо подправить его, чтобы он так делал. Для этого заметим, что $\ffi(\mb F_q)=\mb F_q$. Таким образом у нас возникает автоморфизм $\mb F_q \to \mb F_q$. Он имеет вид $\Frob^{\circ i}$. Тогда на всём поле $L_2$ рассмотрим автоморфизм $\ffi'=\Frob^{\circ -i}$. Тогда композиция $\ffi'\circ \ffi$ и есть подходящий изоморфизм.

Теперь рассмотрим поле из $q^m$ элементов. Тогда в нём есть подполе из $q^l$ элементов только если $nm \di nl$. Но это происходит только если $m \di l$. Такое подполе единственно и автоматически снабжается структурой $\mb F_q$ расширения, так как содержит образ последнего при его вложении в $\mb F_{q^m}$.
\endproof
\ethrm

Покажем одну полезную теорему про многочлены над конечным полем.

\thrm Пусть $f(x)$ -- это неприводимый многочлен из $\mb F_q[x]$. Тогда $x^{q^m}-x \di f(x)$ тогда и только тогда, когда $\deg f(x) | m$.
\proof Пусть $x^{q^m}-x$ делится на $f(x)$. Тогда в поле $\mb F_q^m$ многочлен $f(x)$ имеет корень $\alpha$ (на самом деле там лежат все его корни). Теперь $\mb F_q[\alpha]$ подполе $\mb F_{q^m}$. Но тогда $\deg f(x) = [\mb F_q[\alpha]: \mb F_q] \di m $. 

Обратно, пусть $k=\deg f(x) | m$. Тогда в $\mb F_q^m$ есть подполе $\mb F_{q^k}$. Но такое подполе изоморфно $\mb F_q[x]/f$ и имеет внутри корень $\alpha$ многочлена $f(x)$. Но тогда $f(x)$ и $x^{q^m}-x$ не взаимно просты, откуда следует, что $x^{q^m}-x \di f(x)$, благодаря неприводимости последнего. 
\endproof
\ethrm


43) Лемма про производную. Лемма про эффективное извлечение корня степени $p$.\\
1. $f=\prod g_i^{n_i}$, смотрим $f=g_i^{n_i}g$, берём произв., см. на степень вхождения в $f$ и $f'$. 2. $h'=0 \Leftrightarrow h=g(x^p)$, коэфф. $g$: $a_i$, см. $b_i^p=a_i$ (можно, т.к. Frob), см. $f$ c коэфф. $b_i$ и распиши $f^p=g(x^p)=h$. Для извлечения см. обратный Frob.\\


44) Лемма про разделение на сомножители, чьи неприводимые множители имеют одинаковую степень.\\

Шпаргалка: TODO\\


45) Алгоритм Берлекэмпа.\\

Шпаргалка: TODO\\


46) Вероятностный алгоритм Кантора-Цассенхауза.\\
1. Про кв-ты: $x^{\frac{q^d-1}{2}}=\pm 1$. 2. Как в Б45, такое же $R$, оттуда случ-й $h \rightarrow h^\frac{q^d-1}{2}-1$. Попадём в $\{0, -1, -2\}$. Худший случай: $\mb F_3 \times \mb F_3$. Считаем вероятность получить среди первых двух комп-т квадрат и не квадрат ($p\ge \frac{4}{9}$).\\

%Шпаргалка: TODO\\


\section{
    Коды, исправляющие ошибки...
}

\textbf{Коды, исправляющие ошибки. Минимальное расстояние. Линейные коды. Вычисление минимального расстояния для линейных кодов.}


\section{
 Циклические коды. Эквивалентное описание. Коды БЧХ. Пример.
}

49) Циклические коды. Эквивалентное описание. Коды БЧХ. Пример.


50) Основная теорема про коды БЧХ.


\section{
 Алгоритм декодирования Питерсона-Горенштейна-Цирлера.
}


54) Обратная функция относительно свёртки. Её мультипликативность. Функция Мёбиуса. Формула обращения.\\
1. $f(1)$ -- обратимо, выкидываем $n$, см. $0=\sum_{d|n}f(\frac{n}{d})g(d)$. 2. Инд-ция по $(n, m)$: $g(nm)=-\sum_{d|nm, d<nm}\frac{nm}{d})g(d)=e(n)e(m)+g(n)g(m)$. 3. Пишем дзета-ф-цию, суммируем как БУГП, см. обратную. 4. $f=g*1 \Leftrightarrow g=f*1^{-1}=f\mu$.\\


\section{
 Вероятность встретить два взаимно простых числа.
}

55) Вероятность встретить два взаимно простых числа.


	
\end{document}

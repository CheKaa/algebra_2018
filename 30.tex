\section{
 Алгоритм Кронекера. Сведение для многочленов от нескольких переменных.
}
{\it Шпаргалка.}\\
 1) Перебираем наборы делителей $f(i)$, $0 \le i \le \frac{deg f}{2}$, интерполируем, проверяем. 2) Различным разложениям $f(x_1,\dots,x_n)$ соответствуют различные разложения $f(x, \dots, x^{d^{n-1}})$ для $d$ больших $\max_{i=1}^n \{\deg_{x_i} f\}$. Рассмотреть образ $x^\alpha$.\\

{\bf Алгорим Кронекера}\\
Итак, пусть есть целочисленный многочлен $f(x)$ и мы хотим разложить его на множители. Мы будем искать разложение на целочисленные многочлены, заметим, что хотя бы один из них имеет степень меньшую, чем $[\frac{n}{2}]$. Вспомним о задаче интерполяции. Если $g$ -- искомый делитель $f$, то $g$ определяется своими значениями в $[\frac{n}{2}]+1$ точке, например в точках $0,1,\dots, [\frac{n}{2}]$. Более того, $f(i) \di g(i)$. Таким образом набор $g(0),\dots, g([\frac{n}{2}])$ состоит из делителей $f(0),\dots,f([\frac{n}{2}])$. Найти все такие наборы -- конечный перебор. По каждому набору восстановим $g$ по интерполяции и проверим, является ли он 
делителем $f$.

Сведем задачу разложения многочленов от нескольких переменных к предыдущей.

\thrm Пусть $R$ -- кольцо. Тогда различным разложениям $f(x_1,\dots,x_n)\in R[x_1,\dots,x_n]$   соответствуют различные разложения $\hat{f}=f(x, x^d, x^{d^2}, \dots, x^{d^{n-1}})$ для $d$ больших $\max_{i=1}^n \{\deg_{x_i} f\}$.
\proof Пусть $f=g_1h_1=g_2h_2$ и пусть $g_1\neq g_2$. Покажем, что $\hat{g_1}\neq \hat{g_2}$. Мы рассматриваем отображение $f(x_1, \dots, x_n) \rightarrow 
f(x, x^d,\dots, x^{d^{n-1}})$. Рассмотрим мономом $x^{\alpha}$, где $\alpha$ - это мультииндекс. Он переходит в многочлен $x^{\alpha_1+\alpha_2d+\dots+\alpha_n d^{n-1}}$. По условию все $\alpha_i<d$ как степени при переменных $x_i$. Тогда моном $x^{\alpha_1+\alpha_2d+\dots+\alpha_n d^{n-1}}$ может быть получен только из монома $x^{\alpha}$. Заметим теперь, что $\deg_{x_i} g_j \leq \deg f <d$. Следовательно мономы многочленов $g_j(x)$ так же однозначно восстанавливаются по мономам $\hat{g_j}$.
\endproof
\ethrm

К сожалению, не стоит ожидать взаимооднозначного соответствия между разложениями многочленов $f$ и $\hat{f}$. Например, многочлен $x_2^2$ раскладывается на два множителя одним способом. При $d=3$ его образ есть $x^6$ у которого 3 различных разложения.
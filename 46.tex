\section{
 Вероятностный алгоритм Кантора-Цассенхауза.
}


\thrm[Алгоритмы Кантора-Цассенхауза] Пусть $f(x)\in \mb F_q$ без кратных множителей, $q\neq 2^d$. Тогда существуют вероятностный полиномиальный по $n\log q$  алгоритм раскладывающий $f$ на множители.
\proof
Будем предполагать, что $f(x)$ имеет в качестве неприводимых делителей многочлены степени ровно $d$. Тогда алгебра $R$ имеет вид
$$R= \mb F_q[x]/f(x)\cong \mb F_{q^d}\times \dots \times \mb F_{q^d}.$$ 


Для упрощения обозначений я заменю $q^d$ на $l$. Посмотрим отдельно на один сомножитель $\mb F_{l}$. Заметим, что любой элемент поля $\mb F_l$ удовлетворяет условию, что $x^{\frac{l-1}{2}}$ либо 0, либо 1, либо $-1$. Ноль реализуется только в случае $x=0$, а $1$ и $-1$, если $x$ квадрат и не квадрат соответственно. Из этого стоит пояснить, что, если $x$ не квадрат, то $x^{\frac{l-1}{2}}$ элемент порядка 2 (и следовательно равен $-1$). Действительно, любой элемент $\mb F_l^*$ есть степень некоторого элемента $\alpha$. Тогда элемент квадрат  только если он есть $\alpha^{2d}$. В свою очередь, элемент $\alpha^{2d+1}$ не может быть тривиальным, потому что порядок $\alpha$ чётен. В частности $(\alpha^{2k+1})^{\frac{q-1}{2}}$ с одной стороны имеет порядок либо 2, либо 1 и при этом не тривиален, то есть порядка 2.


Возьмём теперь случайный элемент из алгебры  $R$, при условии, что все неприводимые множители $f$ одной степени, скажем $d$. В последнем случае вся алгебра 
$$R \cong \mb F_{q^d}\times \dots \times \mb F_{q^d}$$
есть произведение одинаковых полей.  Наша задача придумать вероятностный алгоритм находящий делитель нуля в таком произведении. Для этого возьмём случайный элемент $a$ из $R$. Если $a$ делитель нуля всё и так хорошо. Это можно проверить взяв $\Nod(f,a)$, который заодно вычислит делитель $f$. Иначе с вероятностью больше чем  $\frac{1}{2}$ одна из компонент $a$ является квадратом, а ещё одна не является квадратом. Пусть это первая или вторая компоненты. Тогда $a^{\frac{q-1}{2}}$ имеет вид $(1,-1,\dots)$ и $a-1$ является нетривиальным делителем нуля.  
\endproof
\ethrm


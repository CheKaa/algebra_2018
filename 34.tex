\section{
 Результант. Совпадение двух определений (без лемм).
}

шпаргалка

Пусть $f$ и $g$ многочлены степеней $n$ и $m$. Тогда наличие общего множителя (константы не считаются) у $f$ и $g$
$\iff$ существование многочленов $k_1$ и $k_2$, $\deg k_1 <n $, $\deg k_2 < m$, что $f k_2 - g k_1 = 0$

\proof "$\Rightarrow$" 

Если есть общий множитель $h$, то $f = h k_1$ и $g = h k_2$. Тогда $f k_2 - g k_1 = h k_1 k_2 - h k_2 k_1 = 0$

"$\Leftarrow$"

Пусть такие $k_1$ и $k_2$ нашлись. Тогда $f k_2 = g k_1$. Из чего следует, что $f k_2 \di g$. Предположим, 
что $g$ и $f$ взаимно простые. Тогда $k_2 \di g$, что неверно из-за ограничения на степень $k_2$. Значит,
$g$ и $f$ имеют общий множитель.
\endproof


Рассмотрим отображение $K[x]_{\leq m-1}\oplus K[x]_{\leq n-1} \to K[x]_{\leq n+m-1}$, заданное по правилу
$$(k(x),l(x)) \to k(x)f(x)+l(x)g(x).$$ 

Данное отображение вырождено тогда и только тогда, когда есть многочлены маленьких степеней, что $k(x)f(x)=-l(x)g(x)$. 
По доказанному выше, это происходит тогда и только тогда, когда у многочленов $f$ и $g$ есть общий множитель. 
С другой стороны, отображение вырождено, если определитель матрицы данного отображения равен 0.

Возьмем в качестве базиса исходного пространства $1_k, x_k, x^2_k, \ldots, x^{m-1}_k, 1_l, \ldots, x^{n-1}_l$.
Пусть $f(x)=a_0+\dots+a_nx^n$, а $g(x)=b_0+\dots+b_mx^m$.
Тогда матрица оператора будет иметь вид:
$$\begin{pmatrix}
a_0     & 0     & \cdots    &  b_0  & 0     & \cdots    & 0 \\
a_1     & a_0   & \cdots    &  b_1  & b_0   & \cdots    & 0 \\
\vdots  &\vdots & \vdots    & \vdots & \vdots & \vdots  & \vdots \\
0       & 0     & \cdots    & 0     & 0     & \cdots    & b_m \\ 
\end{pmatrix}$$

Транспонируем матрицу, переставим столбцы и, возможно, домножим на (-1):
 
\dfn \textcolor{red}{1} Пусть многочлен $f(x)=a_0+\dots+a_nx^n$, а $g(x)=b_0+\dots+b_mx^m$. Тогда результантом многочленов $f$ и $g$ называется $$Res(f,g)=  \det 
\begin{pmatrix}
a_n & a_{n-1} & \cdots & a_0 & 0 & \cdots & 0 \\
0 & a_n & a_{n-1} & \cdots & a_0 & \cdots & 0 \\
\\
0 & \cdots &  a_n & a_{n-1} & a_{n-2} & \cdots &  a_0 \\
b_m & \cdots & b_1 & b_0 & 0 & \cdots & 0 \\
 \\
0 & \cdots & 0 & b_m & \cdots & b_1 & b_0 
\end{pmatrix}.$$
Эта матрица называется матрицей Сильвестра. 
\edfn

\dfn \textcolor{red}{2} Пусть многочлен $f(x)=a_0+\dots+a_nx^n$, а $g(x)=b_0+\dots+b_mx^m$ из кольца $K[x]$, где $K$ -- поле. 
Пусть так же в поле $K$ имеются разложения $f(x)=a_n\prod(x-x_i)$, а $g(x)=b_m\prod (x-y_j)$. Тогда
$$Res(f,g)=a_n^mb_m^n \prod_{i,j} (x_i-y_j),$$ \edfn
\thrm Определения равны
\ethrm
\proof $\det S$ -- многочлен от коэффициентов $f$ и $g$. Но каждый коэффициент многочлена, кроме старшего, -- симметрический многочлен от корней.
Тогда поделим $p(x_1\dots,x_n,y_1,\dots,y_m)=\det S$ на $(x_i-y_j)$ c остатком как многочлен от $x_i$
$$p=(x_i-y_j)q+r(x_1,\dots, x_{i-1},x_{i+1},\dots,x_n,y_1,\dots,y_m).$$
В остатке стоит многочлен $r$ степени 0 по $x_i$, то есть от $x_i$ не зависящий. Подставим $x_i = y_j$, $Res(f, g) = \det S = p$ станут равны нулю (из того, как мы строили результант). Тогда и $r=0$, то есть $\det S \di (x_i-y_j)$. Так как все $(x_i - y_j)$ неприводимы (достаточно?) в $R[y_0, \ldots, y_m, x_0, \ldots, x_n]$, то $\det S \di \prod (x_i-y_j)$. 



\lm  $$a_n^mb_m^n \prod_{i,j} (x_i-y_j)=(-1)^{mn}b_m^n \prod f(y_j)=a_n^m \prod g(x_i).$$ 
\elm

Теперь видно, что результант есть однородный многочлен степени $m$ по координатам $a_i$ и степени $n$ по координатам $b_j$. Но резьтант обладает в точности тем же свойством. Значит они равны с точностью до обратимого множителя. Для этого найдём коэффициент при $a_n^m$ у обоих выражений. В результанте этот коэффициент идёт с множителем $b_0^m$. В выражении  $(-1)^{mn}b_m^n \prod f(y_j)$ это $(-1)^{mn}b_m^n \prod_{j=1}^m y_j^n$. Оба эти выражения совпадают.

\endproof

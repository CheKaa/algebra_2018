\section{
 Основная теорема про конечные поля.
}

\thrm Существует и единственно (с точностью до изоморфизма) поле из $p^n$ элементов. Обозначается $\mb F_{p^n}$.

Поле разложения $x^{p^n} - x \to$ подполе из $p^n$ элементов. Взять образующий группы, найти мин. многочлен, найти его корень в другом поле (через делимость). И проверить на изоморфизм ``образующий группы в корень'' -- техника.


\proof[{\color{red!80!black} Доказательство теоремы. Существование}]
	
	Возьмём $\mb Z/p$ и многочлен $x^{p^n} - x$ над ним. Рассмотрим такое расширение, где этот многочлен раскладывается на линейные множители. Мы знаем, что такое есть из прошлого семестра (мы умеем последовательно строить расширения, где многочлен имеет хотя бы один корень).

	Рассмотрим такое поле $L$. В нём есть подполе элементов, удовлетворяющих $x^{p^n} - x = 0$ (по лемме). Но таких элементов ровно $p^n$: их не больше $p$, так как корней многочлен в поле не больше $p$, но и не меньше, так как у этого многочлена нет кратных корней (производная).


\proof[{\color{red!80!black} Доказательство теоремы. Единственность}]

	Возьмём два поля $K, L$ такие, что $|K| = |L| = p^n$.

	Рассмотрим $\ffi$ -- образующий циклической группы $K^*$. Понятно тогда, что добавив $\ffi$ к $\mb Z/p$ получится $K$, ведь $\ffi$ даже умножением уже всё порождает. То есть $K \cong (\mb Z/p[x])/m_{\ffi}$, где $m_{\ffi}$ -- минимальный многочлен для $\ffi$.

	Теперь рассмотрим $m_{\ffi}$ над $L$. У него есть корни, так как $x^{p^n} - x$ делится на $m_{\ffi}$ над обоими полями (делимость не портится при переходе между полями, ведь делимость -- алгоритм Евклида, а коэффициенты вообще из базового поля). 

	То есть у $m_{\ffi}$ есть корень в $L$ -- $\xi$, тогда $f : (\mb Z/p[x]) \to L,\ x \mapsto \xi$ -- просто гомоморфизм подстановки, но он пропускается через $(\mb Z/p[x])/m_{\ffi}$, так как кратные $m_{\ffi}$ уж точно зануляются. Тогда гомоморфизм $K = (\mb Z/p[x])/m_{\ffi} \to L$ -- изоморфизм: инъективность следует из тривиальности ядра (если ядро не тривиально, то оно всё $K$ (промежуточных нет, т. к. поля), но мы понимаем, что отображение не нулевое), сюрьективность из инъективности и Дирихле.

\ethrm
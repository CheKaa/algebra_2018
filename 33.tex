\section{
 Целые алгебраические элементы. Замкнутость относительно операций.
}

{\bf Целый алгебраический элемент}\\
Элемент $\alpha \in \mb C$ называется целым алгебраическим или просто целым, если существует такой многочлен $f(x)\in \mb Z[x]$, что $f(\alpha)=0$ и при этом $f\neq 0$ и старший коэффициент $f$ равен 1.\\


{\bf Теорема.} Сумма целых элементов цела и произведение целых элементов цело. \\
{\bf Доказательство}\\
Пусть $\alpha$ и $\beta$ -- целые, а $f(x)=a_0+\dots+x^n$ и $g(x)=b_0+\dots+x^m$ -- обнуляющие их целочисленные многочлены со старшим коэффициентом, равным единице. Пусть $\alpha=\alpha_1,\dots,\alpha_n$ и $\beta=\beta_1,\dots,\beta_m$. Тогда рассмотрим многочлен
$$\prod_{i,j} (x -(\alpha_i+\beta_j))$$
его корнем является $\alpha+\beta$ и он имеет старший коэффициент 1. Осталось показать, что его коэффициенты целые. Представим на секунду, что $\alpha_i$ и $\beta_j$ -- это независимые переменные. В самом конце подставим в них настоящие значения. Тогда этот многочлен симметричен по $\alpha_i$ в кольце $\mb Z[\beta_1,\dots,\beta_m]$. Следовательно его коэффициенты выражаются через $a_i=\pm \sigma_{n-i}(\alpha_1,\dots,\alpha_n)$, $\beta_1,\dots,\beta_m$ и целые числа. Но это выражение симметрично и по $\beta_j$. Откуда коэффициенты этого многочлена выражаются через $\sigma_{n-i}(\alpha_1,\dots,\alpha_n)$,  $\sigma_{m-j}(\beta_1,\dots,\beta_m)$ и целые числа. Теперь подставляя настоящие $\alpha_i$, $\beta_j$ получаем, что это многочлен с целыми коэффициентами, так как $a_i$ и $b_j$ целые по условию.

Аналогично, рассмотрим многочлен 
$$\prod_{i,j} (x-\alpha_i\beta_j).$$
Представим на секунду, что $\alpha_i$ и $\beta_j$ -- это независимые переменные. В самом конце подставим в них настоящие значения. Тогда этот многочлен симметричен по $\alpha_i$ в кольце $\mb Z[\beta_1,\dots,\beta_m]$. Следовательно его коэффициенты выражаются через $a_i=\pm \sigma_{n-i}(\alpha_1,\dots,\alpha_n)$, $\beta_1,\dots,\beta_m$ и целые числа. Но это выражение симметрично и по $\beta_j$. Откуда коэффициенты этого многочлена выражаются через $\sigma_{n-i}(\alpha_1,\dots,\alpha_n)$,  $\sigma_{m-j}(\beta_1,\dots,\beta_m)$ и целые числа. Теперь подставляя настоящие $\alpha_i$, $\beta_j$ получаем, что это многочлен с целыми коэффициентами, так как $a_i$ и $b_j$ целые по условию.

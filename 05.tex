\section{
 Оценка на собственные числа ограничения. Оценка на след.
}

{\bf5)} Оценка на собственные числа ограничения. Оценка на след.

Шпаргалка: TODO

\thmm{Оценка на собственные числа ограничения}

Пусть $q(x)$~--- квадратичная форма на евклидовом пространстве $V$; $q(x) = x^TAx$. $U \le V$. Рассмотрим сужение $q$ на $U$: $q|_U$. Тогда этой форме соответствует оператор $B$ такой, что $q|_U = x^TBx$. Обозначим собственные числа $A$ как $\lambda_i$, при чём $\lambda_i \ge \lambda_{i+1}$. Аналогично собственные числа $B$~--- $\mu_j$; $\mu_j\ge \mu_{j + 1}$. Тогда выполняются неравенства:
$$
\lambda_{i+n-m}\le\mu_i\le\lambda_i
$$

\proofname

$\mu_i\le\lambda_i$

Пусть $dim\,U = m,\ dim\,V = n$. Тогда по теореме Куранта-Фишера:
$$
\lambda_i = \max\limits_{L \le V,\,dim\,L = i} \min\limits_{x\in L} q(x) 
$$

Такое же равенство для $\mu_i$:
$$
\mu_i = \max\limits_{T \le U,\,dim\,T = i} \min\limits_{x\in T} q(x) 
$$

$T\le U\le V\Rightarrow\ T\le V$. Значит, в равенстве для $\lambda_i$ максимум берётся по всем подпространствам, которые учтены в $\mu_i$ и ещё каким-то $\Rightarrow$ $\lambda_i \ge \mu_i$. Что и требовалось.

$\mu_i \ge \lambda_{i + n - m}$

Применим второе равенство из теоремы Куранта-Фишера.

$$
\lambda_{i+n-m} = \min\limits_{L \le V,\,dim\,L = n - (i + n - m) + 1 = m - i + 1} \max\limits_{x\in L} q(x) 
$$

$$
\mu_i = \min\limits_{T \le U,\,dim\,T = m - i + 1} \max\limits_{x\in T} q(x) 
$$

По тем же соображениям $T \le V$. В равенстве для $\lambda_i$ минимум берётся по всем подпространствам, которые учтены в $\mu_i$ и ещё каким-то $\Rightarrow$ $\lambda_{i + n - m}\le \mu_i$. Что и требовалось.

\defn{След квадратичной формы}

Пусть $q$~--- квадратичная форма на евклидовом пространстве $V$. $u_i$~--- ортонормированный базис $V$. Тогда определим след $q$ следующим образом:
$$
Tr\,q=\sum\limits_{i=1}^{dim\,V} q(u_i)
$$

\thmm{Это действительно след!}

Если в базисе $u_i$ форме $q(x)=x^TAx$ соответствует симметричная матрица $A$, то $Tr\,q=Tr\,A$. И след квадратичной формы не зависит от выбора ортонормированного базиса.

\proofname

Посчитаем след матрицы $A$ в базисе $u_i$.

Как вычислить значение $i$-го диагонального элемента матрицы? Возьмём $v_i=(0,\ldots,1,\ldots 0)$ (единица на $v$-ой позиции). Посчитаем $v_iAv_i^T$. Получим $A_{i, i}$. (Упражнение: убедиться, что это действительно так.)

Но $q(u_i) = (0,\ldots,1,\ldots,0)A(0,\ldots,1,\ldots,0)^T = v_iAv_i^T$.

След не зависит от выбора базиса, так как при переходе к новому базису у нас получится $Tr\,CAC^{-1}$, а последнее равно $Tr\,A$, так как $Tr(AB)=Tr(BA)$ (прошлый семестр) и $C$-шки умрут.

\thmm{Оценка на след}

Пусть $U\le V$, где $V$~--- евклидово, $dim\,V = n,\ dim\,U = m$. $q(x)$~--- квадратичная форма на $V$. $q(x)\ge 0$ на всех $x$. Тогда:
$$
Tr(q)\ge Tr(q|_U)
$$

\proofname

След оператора~--- сумма его собственных чисел. По неравенству $\lambda_i\ge\mu_i$ и неотрицательности $q$ получаем $\sum\limits_{i=1}^n \lambda_i \ge \sum\limits_{i=1}^m \mu_i$, а это как раз следы соответствующих операторов (а они равны следам квадратичных форм).

\rem А в будущем потребуется утверждение $\sum\limits_{i = 1}^m \lambda_i \ge \sum\limits_{i = 1}^m \mu_i$ (то есть не для $V$, а для подпространства $V$, натянутого на первые $m$ собственных векторов). Оно верно и без требования $q(x)\ge 0$.

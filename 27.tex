\section{
 Лемма Гаусса. Содержание многочлена... %не надо писать формулы и \ldots
}

\textbf{Лемма Гаусса. Содержание многочлена. Делимость в $Q(R)[x]$ и в $R[x]$.}

\textbf{Необработанная версия из конспекта Константина Михайловича}


Настала пора вернуться к кольцу многочленов. Но на этот раз мы поговорим подробно о кольце многочленов от $n$ переменных. Мы знаем, что кольцо многочленов $K[x]$ над полем $K$ является областью главных идеалов. Есть ли надежда сказать тоже самое про многочлены от двух переменных? 

Ответ на этот вопрос даёт следующий пример: рассмотрим идеал $\lan x,y \ran$ в кольце $K[x,y]$. Это максимальный идеал. Но как мы знаем, этот идеал нельзя породить одним элементом (см. отступление про модули в прошлом семестре). 

Однако мы интуитивно представляем то, что разложение многочленов на множители однозначно. Напомню, что математически такое свойство называлось факториальностью. Итак, наша ближайшая цель поговорить о факториальности колец многочленов. Однако, для приложений к теории чисел нам будет необходимо разработать теорию в достаточной общности, чтобы применить её и над $\mb Z$, а не только над полем. Заметим, что $\mb Z$ и поле $K$ являются факториальными кольцами. Мы покажем, что факториальность кольца влечёт факториальность кольца многочленов над ним, что полностью ответит на наши вопросы.



\textbf{Многочлены над факториальным кольцом}

Наша задача обсудить, что происходит с кольцом многочленов от многих переменных.


\lm[Гаусс] Пусть $R$ -- факториальное кольцо. Тогда любой простой элемент $p$ из $R$ остаётся простым в $R[x]$.
\proof
Теоретически удобно воспользоваться следующим соображением: чтобы показать, что элемент прост надо показать, что идеал $(p)$ в $R[x]$ прост, а для этого необходимо и достаточно установить, что $R[x]/(p)$ есть область целостности. Чему же равно $R[x]/(p)$? Я утверждаю, что оно равно $R/p[x]$. Действительно, из $R[x]$ есть отображение в $R/(p)[x]$, которое берёт все коэффициенты по модулю $p$. Очевидно, в его ядре лежат многочлены, все коэффициенты которых делятся на $p$, то есть многочлены кратные $p$ в $R[x]$. Но ровно они и образуют идеал $(p)$. Осталось заметить, что кольцо $R/p$ и вслед за ним кольцо $R/p[x]$ являются областями целостности.

У этого доказательства есть другая, более элементарная реинкарнация. А именно, формально нам надо доказать, что если произведение двух многочленов $f(x)g(x)$ делится на $p$ (то есть все коэффициенты кратны $p$), то тогда какой-то из них делится на $p$. Пусть это не так. Возьмём тогда у $f$ и у $g$ самые младшие коэффициенты $a_i$ и  $b_j$, которые не делятся на $p$. Тогда посмотрим на коэффициент с номером $i+j$  в произведении. Он имеет вид $c_{i+j}= a_ib_j + \sum_{k \neq i} a_k b_{i+j -k}$. Я утверждаю, что $c_{i+j}$ не делится на $p$. Для этого заметим, что любое слагаемое в сумме делится на $p$, так как либо $k<i$ и тогда $a_i \di p$, либо $k>i$, то есть $i+j-k<j$ и следовательно $b_j \di p$. Противоречие с тем, что $c_{i+j}$ должен делиться на $p$.   
\endproof
\elm

\bupr Поясните, почему оба доказательства одинаковы.
\eupr

\dfn Пусть $f(x)$ -- многочлен над факториальным кольцом $R$. Тогда содержанием $f$ называется $\cnt(f)=\Nod (a_i)$, где $a_i$ коэффициенты $f$. 
\edfn

Тут есть некоторая вольность -- надо помнить, что наибольший общий делитель определём с точностью до обратимых множителей. Следующее следствие тоже называют леммой Гаусса.

\crl Если $f(x)=g(x)h(x)$, где $f,g,h \in R[x]$, то $\cnt(f)=\cnt(g)\cnt(h)$
\proof Для начала, упростим задачу, то есть сведём задачу к случаю $\cnt g= \cnt h =1$. Для этого надо рассмотреть многочлены $\frac{g}{\cnt g}$ и $\frac{h}{\cnt h}$. Их произведение есть $\frac{f}{\cnt{g}\cnt{h}}$ имеет содержание $\frac{\cnt f}{\cnt g \cnt h}$ и если показать его единичность, то мы добьёмся требуемого. Итак считаем, что $\cnt g= \cnt h=1$. Если $\cnt f$ не обратим, то $\cnt f \di p$, где $p$ простой элемент из $R$. Но тогда один из $g$ или $h$ делится на $p$ благодаря его простоте. 
\endproof
\ecrl


\lm Пусть для многочлена $f(x) \in R[x]$  имеет место разложение $f(x)=g(x)h(x)$, где  $g(x)h(x) \in Q(R)[x]$. Тогда существуют такая константа $c \in Q(R)$, что $cg \in R[x]$ и $c^{-1}h \in R[x]$, что означает, что $f(x)=cg(x)c^{-1}h(x)$ -- есть произведение двух многочленов из $R[x]$ пропорциональных исходным.
\proof
Рассмотрим несократимую запись для коэффициентов $g$ и $h$ и обозначим наибольший общий делитель их знаменателей за $d_1$ и $d_2$ соответственно. Тогда $d_1g$ и $d_2h$ лежат в $R[x]$. Имеет место равенство $d_1 d_2\cnt f = \cnt(d_1 g) \cnt(d_2h)$. Таким образом, правая часть делится на $d_1d_2$. На самом деле $\cnt(d_1 g) \di d_2$ так как $d_2$ и $\cnt(d_2h)$ взаимно просты. Аналогично $\cnt(d_2h) \di d_1$. Осталось взять в качестве $c= \frac{d_1}{d_2}$.

\endproof
\elm

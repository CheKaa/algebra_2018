\section{
 Степенные суммы. Тождество Ньютона.
}

Шпаргалка: $0=(-1)^n n\sigma_n + \sum_{k=0}^{n-1}(-1)^k \sigma_k s_{n-k}$, в многочлен подставим корни, просуммируем по всем корням, отдельно случаи $k < n$ - добавим нулевые переменные, $k > n$ - занулим не входящие в моном переменные\\
\\

{\bf Степенная сумма}\\
$s_k(x_1, \dots, x_n)=x_1^k+\dots+x_n^k$\\
\\
{\bf Элементарная симметрическая функция}\\
Функция $\sigma_k(x_1,\dots, x_n)= \sum_{1\leq i_1<\dots<i_k\leq n}x_{i_1}\dots x_{i_k}$ называется элементарной однородной степени $k$ симметрической функцией от переменных $x_1,\dots, x_n$. Если многочлен $p(x)=(x-x_1)\dots(x-x_n)=x^n+a_{n-1}x^{n-1}+\dots+a_0$, то $a_i=(-1)^{n-i}\sigma_{n-i}(x_1,\dots, x_n)$.\\
\\
{\bf Тождество Ньютона}\\
Степенные суммы и элементарные симметрические многочлены связаны тождествами
$$0=(-1)^n n\sigma_n + \sum_{k=0}^{n-1}(-1)^k \sigma_k s_{n-k}$$\\
\\
{\bf Доказательство}\\
1. Докажем равенство, если число переменных равно n\\
Рассмотрим равенство $$(x-x_1)\dots(x-x_n)=x^n+\sum (-1)^{n-i}\sigma_{n-i} x^i.$$
Равенство верно по определению $\sigma_{n - i}$\\
Подставим в это равенство $x=x_j$. Получим $$0=x_j^n+\sum (-1)^{n-i}\sigma_{n-i} x_j^i.$$
Просуммируем по всем $j$. Получим 
$$0=s_n+\sum_{i\neq 0} (-1)^{n-i}\sigma_{n-i} s_i + (-1)^nn\sigma_n$$
Так как $\sum_j x_j^i = s_i$\\
Это доказывает равенство, когда число переменных равно номеру $\sigma_n$. Подставив переменные $x_{k+1},\dots, x_n$ равные 0 в это равенство получим его для $k$ переменных $k<n$. \\
Теперь  предположим, что $k>n$. Проверим, что справа и слева однинаковые мономы входят с одинаковым коэффициентом.
Заметим, что в каждом мономе заведомо участвует не более $n$ различных переменных так степень каждого монома ровно $n$. Пусть мы хотим проверить наличие справа и слева одинакового числа мономов в записи которых участвуют  переменные $x_{i_1},\dots, x_{i_n}$. Подставим вместо всех остальных переменных 0. Понятно, что с искомым мономом ничего не произойдёт. С другой стороны после такой подстановки и переобозначения переменных мы приходим к уже доказанному равенству, когда $k=n$.

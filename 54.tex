\section{
 Обратная функция относительно свёртки... %не заменять на ldots
}

\textbf{Обратная функция относительно свёртки. Её мультипликативность. Функция Мёбиуса. Формула обращения.}


\thrm Пусть $f\colon \mb N \to R$, тогда $f$ -- обратима тогда и только тогда, когда $f(1)\in R^*$ и обратная функция $g$ задаётся формулой
$$g(n)=\frac{-1}{f(1)}\sum_{d|n,\, d<n} f\left(\frac{n}{d}\right)g(d).$$
\proof Очевидно, что обратимость $f(1)$ необходима. Для того чтобы $f$ и $g$ были обратны необходимо и достаточно, чтобы $g(1)=\frac{1}{f(1)}$ и для каждого $n>1$ было выполнено
$$0=\sum_{d |n} f\left(\frac{n}{d}\right)g(d).$$
Осталось перенести последнее слагаемое в левую часть и поделить на $-f(1)$.
\endproof
\ethrm

\lm Обратная к мультипликативной функции снова мультипликативна.
\proof Проще всего это понять через $L$-функции. Действительно, каждый сомножитель в разложении в произведение имеет вид $1+\frac{f(p)}{p^s}+\frac{f(p^2)}{p^{2s}}+\dots$ есть ряд от $y=p^s$ начинающийся с единицы. Следовательно, обратный элемент к нему тоже ряд по $y$. Итого обратная функция тоже раскладывается в аналогичное произведение. 
Но можно и воспользоваться доказанной формулой про обратную функцию. Рассуждая по индукции, предположим, что для всех пар $n_1,m_1$ меньших $n,m>1$ уже показана мультипликативность. Тогда 
$$g(nm)=-\sum_{\substack{d|nm\\ d<nm}} f\left(\frac{nm}{d}\right)g(d)=-\sum_{d_1|n}\sum_{d_2|m} f\left(\frac{n}{d_1}\right)g(d_1)f\left(\frac{m}{d_2}\right)g(d_2)+g(n)g(m)=g(n)g(m),$$
так как первое слагаемое есть произведение $e(n)e(m)$. 
\endproof
\elm

Наша задача сейчас описать обратную к функции $1(n)$. Для этого вспомним, что $$\zeta(s)=\prod\left(1+\frac{1}{p^s}+\frac{1}{p^{2s}}+\dots\right)= \prod \frac{1}{1-p^{-s}}.$$
Тогда $\zeta(s)^{-1}$ имеет вид
$$\zeta(s)^{-1}=\prod\left(1-\frac{1}{p^{s}}\right).$$
Если раскрыть скобки, то в получившейся сумме не нулевые слагаемые будут только для бесквадратных $n$, а коэффициент перед ними будет $(-1)^k$, где $k$ -- число простых сомножителей.

\dfn[Функция Мёбиуса] Определим функцию $\mu(n)$ по следующему правилу
$$\mu(n)=\begin{cases}
0, \text{ если существует простое $p$, что $p^2|n$}\\
(-1)^k, \text{ если $n=p_1\dots p_k$}
\end{cases}$$
\edfn

Таким образом функция Мёбиуса обратна к $1(n)$. Это означает, что 
\crl Равенство $f(n)=\sum_{d|n} g(d)$ верно для всех $n$, тогда и только тогда, когда верно $g(n)=\sum_{d|n} \mu(d)f(\frac{n}{d})$.
\proof По условию $f=g*1$. Это происходит тогда и только тогда, когда $g=f*1^{-1}=f*\mu$.
\endproof
\ecrl


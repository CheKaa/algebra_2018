\section{
 Тензорная алгебра. Свёртка и след.
}

\textbf{Необработанная версия из конспекта Константина Михайловича}

\dfn Рассмотрим пространства $V^{p,q}$ и $V^{p',q'}$. Тогда имеет место билинейное отображение $$V^{p,q}\times V^{p',q'} \to V^{p+p',q+q'},$$
заданное правилом 
$$(v^1\otimes\dots\otimes v^p\otimes u_1\otimes\dots \otimes u_q,\hat{v}^1\otimes\dots\otimes \hat{v}^{p'}\otimes \hat{u}_1\otimes\dots \otimes \hat{u}_{q'}) \to v^1\otimes\dots\otimes v^p\otimes \hat{v}^1\otimes\dots\otimes \hat{v}^{p'}\otimes u_1\otimes\dots \otimes u_q \otimes \hat{u}_1\otimes\dots \otimes \hat{u}_{q'}.$$
Такое умножение задаёт структуру ассоциативной алгебры на пространстве 
$$T(V)=\bigoplus_{p,q\geq 0} V^{p,q},$$
которое называется тензорной алгеброй пространства $V$.
\edfn

Посмотрим теперь на пространство $V^*\otimes V$. Определим из него каноническое, то есть не зависящее от выбора базиса отображение в $K$. Действительно элементы этого пространства есть суммы тензоров вида $f\otimes v$. Сопоставим каждому такому тензору 
$$(f,v) \to f(v).$$
Такое сопоставление продолжается до линейного отображения $$Conv \colon V^*\otimes V \to K.$$
Что это за отображение в координатах? Тензор типа $(1,1)$ записывается в базисе как $T=\sum_{i,j} T_j^i e^j\otimes e_i$. Тогда $$Conv(T)=\sum_{i,j} T_j^i e^j(e_i)=\sum_i T^i_i.$$
Если вспомнить, что пространство $V^*\otimes V \cong \Hom(V,V)$, то указанное отображение становится просто отображением следа. Это ещё один способ доказать инвариантность следа. Обозначение $Conv$ взято благодаря слову <<convolution>>, то есть свёртка. Сейчас мы проделаем аналогичную конструкцию в более общем случае. 

\dfn Рассмотрим пространство $V^{p,q}$ и пару индексов $j\leq p$ и $i\leq q$. Тогда свёрткой по индексам $i,j$ называется линейное отображение 
$$Conv_{i,j} \colon V^{p,q}\to V^{p-1,q-1},$$
заданное по правилу 
$$ \dots \otimes f^j\otimes \dots \otimes v_i \otimes \dots \to f^j(v_i)\cdot f^1\otimes \dots.$$
Это одно из самых часто используемых понятий про тензоры. В координатном виде это просто сумма по совпадающим индексам в позиции $j$ и $i$. Понятно, что свёртку можно делать по одинаковым по размеру упорядоченным группам координат. Я не буду про это говорить дополнительно.
\edfn


\section{
 Замена переменной – общий случай.
}

\textbf{Необработанная версия из конспекта Константина Михайловича}

Теперь мы можем разобраться с тензорами общего вида:

\thrm Пусть $e_1,\dots,e_n$ старый базис $V$, а $\hat{e}_1,\dots,\hat{e}_n$ -- новый. Пусть $C$ -- матрица перехода из старого базиса в новый, а $D={C^{\top}}^{-1}$. Тогда координаты тензора $T$ в базисе $\hat{e}$ выражаются через старые координаты следующим образом:
$$\hat{T}_{j_1,\dots,j_p}^{i_1,\dots,i_q}=\sum_{\substack{i'_1,\dots,i'_q \in \ovl{1,n}\\ j'_1,\dots,j'_p \in \ovl{1,n}}} \,\,
\prod_{t\in \ovl{1,p}} D_{j_t,j'_t} \prod_{s\in \ovl{1,q}} C_{i_s,i'_s}  \,\,T_{j'_1,\dots,j'_p}^{i'_1,\dots,i'_q}.$$
\proof Обозначим за $e^{j_1,\dots,j_p}_{i_1,\dots,i_q}$ тензор $e^{j_1}\otimes \dots \otimes e^{j_p} \otimes e_{i_1}\otimes \dots \otimes e_{i_q}$. Рассмотрим тензор
$$ T= \sum_{\substack{i'_1,\dots,i'_q \in \ovl{1,n}\\ j'_1,\dots,j'_p \in \ovl{1,n}}} T_{j'_1,\dots,j'_p}^{i'_1,\dots,i'_q} e^{j'_1,\dots,j'_p}_{i'_1,\dots,i'_q}$$ 
и заменим $e_i=\sum_{j=1}^nC_{ji}\hat{e}_j$ и $e^i=\sum D_{ji}\hat{e}^j$. Получится такая сумма:
$$ T= \sum_{\substack{i'_1,\dots,i'_q \in \ovl{1,n}\\ j'_1,\dots,j'_p \in \ovl{1,n}}} T_{j'_1,\dots,j'_p}^{i'_1,\dots,i'_q} \sum_{\substack{i_1,\dots,i_q \in \ovl{1,n}\\ j_1,\dots,j_p \in \ovl{1,n}} } D_{j_1j'_1}\dots D_{j_pj'_p} C_{i_1i'_1}\dots C_{i_qi'_q} \hat{e}^{j_1,\dots,j_p}_{i_1,\dots,i_q}$$
Осталось поменять суммирование местами.
\endproof
\ethrm

Важность тензоров в теоретической физике обуславливается тем, что практически все физические объекты -- это тензоры. Точнее: с точки зрения теории относительности пространство-время это некоторое четырёхмерное многообразие $M$ (в двумерной ситуации подошла бы обычная сфера или тор). С каждой точкой $x$ этого многообразия связано касательное пространство в этой точке -- некоторое четырёхмерное пространство $T_x$. Представим себе, что в каждой точке пространства задана плотность вещества (на самом деле не так, но допустим) -- это даёт вам функцию $f \colon M \to \mb R$ -- скаляр в каждой точке, то есть тензор типа $(0,0)$. 

Направление движения материи можно задать взяв в каждой точке касательный вектор, то есть тензор ранга $(0,1)$ на $T_x$. Дальше, у каждого такого вектора можно считать его <<длину>> и углы между векторами. Для этого надо задать для каждой точки $x$ билинейную форму на касательном пространстве, то есть элемент $T_x^{(2,0)}$. И т.д. Чаще всего такие объекты называют тензорными полями, если хочется подчеркнуть, что в разных точках это тензор вообще говоря на разных пространствах.

Важно, что уравнения в физике не должны зависеть от выбора координат. Можно, конечно, писать какие-то уравнения при помощи координат тензоров и каждый раз проверять, что выбрав новые координаты уравнение будет того же вида. Однако, чем сложнее наука тем сложнее становятся проверки. Становится важно работать с тензорами не рассматривая их координаты. Для этого мы обсудим две операции с тензорами, которые легко можно понять не используя координаты. Начнём с самой простой -- умножение тензоров.


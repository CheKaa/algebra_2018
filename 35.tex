\section{
 Леммы про результант. Дискриминант, его смысл. Вычисление через результант.
}

\lm  $$a_n^mb_m^n \prod_{i,j} (x_i-y_j)=(-1)^{mn}b_m^n \prod f(y_j)=a_n^m \prod g(x_i).$$ 
\elm

\proof
$$a_n^mb_m^n \prod_{i,j} (x_i-y_j) = b_m^n \prod_j (a_n \prod_i (y_j - x_i)) = (-1)^{mn}b_m^n \prod f(y_j)$$
\endproof

Продолжение доказательства из предыдущего билета:

Подставляем формулы вида ($f(x)=a_0+\dots+a_nx^n$) в лемму, понимаем, что результант есть однородный многочлен степени $m$ из $R[a_0, \ldots, a_n]$ и степени $n$ из
$R[b_0, \ldots, b_n]$. Значит, и $\det S$ и другая формула для результанта -- однородные многочлены степени $n+m$ над коэффициентами $a_i$ и $b_j$. 
Заметим, что делимость, доказанная ранее, у нас осталась и для такого рассмотрения многочленов.

\proof обозначим вторую формулу как $Res'$.

$\det S = Res' \cdot h$. (Как многочлены от корней). Заметим, что $h$ тоже симметрический многочлен от корней. Поэтому выражается через коэффициенты многочленов.
Значит $\det S \di Res'$ в $R[a_0, \ldots, a_n, b_0, \ldots, b_n]$.
\endproof

Значит, они равны с точностью до обратимого множителя. Для этого найдём коэффициент при мономе $a_n^m b_0^n$ у обоих выражений. 
В результанте данный коеффициент равен $1$. Получаем моном $a_n^m b_0^n$ из выражения $(-1)^{mn}b_m^n \prod f(y_j)$:
$$(-1)^{mn}b_m^n a_n^m \prod_{j=1}^m y_j^n = (-1)^{mn} b_m^n a_n^m (-1)^{mn} \sigma_m^n = a_n^m b_m^n \sigma_m^n = a_n^m b_0^n$$
Коэффициент равен 1.

\hfill
\hfill

\bupr Кроме того, если $f=gq+r$, где $\deg r=k$, то 
 $$Res(f,g)=(-1)^{(n-k)m}b_m^{n-k} Res(r,g).$$
 \proof
    Доказывается подставлением первой части леммы вместо $Res(r, g)$.
 \endproof
 \eupr

\dfn Дискриминантом многочлена $f=a_0+\dots +a_nx^n$ называется выражение 
$$D(f)=a_n^{2n-2}\prod_{i < j} (x_i-x_j)^2.$$
\edfn

\lm Имеет место равенство $Res(f,f')=(-1)^{\frac{n(n-1)}{2}} a_n D(f).$
\proof Прежде всего вспомним (дифференциируем $a_n \prod (x - x_i)$), что если $x_1,\dots,x_n$ -- корни многочлена $f(x)$, то $$f'(x_i)=a_n\prod_{j < i}(x_i-x_j).$$
Посчитаем теперь пользуясь леммой из начала билета.
$$Res(f,f')= a_n^{n-1}\prod_{i=1}^n f'(x_i)=a_n^{2n-1} \prod_{i=1}^n \prod_{j\neq i} (x_i-x_j).$$
Осталось вынести знак из половины скобок слева.
\endproof
\elm

Дискриминант даёт ответ на вопрос, когда многочлен не имеет кратных корней по модулю $p$. 
Действительно это происходит только тогда, когда $D(f)\ndi p$. 
Заметим, что это условие может нарушаться только в конечном числе $p$, если
$D(f)\neq 0$. Т аким образом либо у многочлена есть кратный корень, либо у него
нет кратных корней для почти всех $p$.  Это обосновывает, что для применения
леммы Гензеля для подъёма всегда можно выбрать подходящее простое, если
многочлен $f$ был бесквадратный.


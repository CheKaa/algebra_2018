\section{
		Определение алгебры кватернионов... %не заменять на ldots
	}

\textbf{ Определение алгебры кватернионов. Векторное произведение. Сопряжённый кватернион. Норма кватерниона. Мультипликативность нормы. Сумма четырёх квадратов.}\\
\\
Наша цель сейчас рассказать про геометрию трехмерного пространства используя при этом определённые алгебраические конструкции. А именно, ещё в XIX веке Уильям Роуэн Гамильтон стал искать аналогичную комплексным числам алгебраическую систему на трёхмерном пространстве.  
Однако, подходящий аналог удалось найти только в четырёхмерной ситуации.

Рассмотрим подпространство в алгебре матриц $M_2(\mb C)$ вида
$$\mb H = \left\{\pmat \alpha & \beta \\ -\ovl{\beta} & \ovl{\alpha} \epmat \right\}.$$
Базис этого пространства, как вещественного векторного пространства, состоит из матриц 
$$ 1=\pmat 1 & 0 \\ 0& 1 \epmat, i= \pmat i & 0 \\ 0& -i \epmat, j=\pmat 0& 1 \\ -1 & 0 \epmat, k=\pmat 0 & i \\ i & 0\epmat. $$ 
\\
Для этого достаточно показать их линейную независимость и замкнутость относительно сложения. Второе очевидно, для первого, например, заметим, что никакая нетривиальная линейная комбинация первых двух матриц не может занулить оба элемента на главной диагонали, а две других матрицы не содержат элементов на главной диагонали. Аналогично для двух правых матриц.\\ 
\\ 
Покажем, что это вещественная подалгебра в $M_2(\mb C)$ и следовательно ассоциативное кольцо. \\
\\
Ассоциативность умножения следует из определения в виде подпространства матриц.\\
\\
Далее достаточно показать, что произведение базисных снова лежит в $\mb H$. Имеем $$i^2=j^2=k^2=-1 \text{ и } ij=k=-ji,$$, проверяется прямым умножением матриц. Отсюда получаем $$ik = i(k)= i(ij)=(ii)j = -j = j (-1) = j(ii) = (ji)i =(-k)i =-ki \text{ и } jk=-jjk=-i=-kj.$$ Таким образом $\mb H$ образует ассоциативную алгебру размерности 4 над $\mb R$.

\dfn[Алгебра кватернионов] $\mb H$ называется алгеброй кватернионов. 
\edfn

В дальнейшем будем рассматривать кватернионы как формальные суммы $a + bi + cj + dk$ с верными доказанными выше тождествами.\\
\\
\dfn[Вещественная и мнимая часть] Вещественная или скалярная часть кватерниона $\Re x=a$, мнимая или векторная часть $v=\Im x= bi+cj+dk$.\\
\edfn

\dfn[Векторное произведение] Пусть $u = (a,b,c),v = (a',b',c') \in \mb R^3$ два вектора. Тогда их векторным произведением называется вектор $[u,v]$, задаваемый по формуле $$[u,v]= (bc'-cb')i + (ca'-ac')j + (ab'- ba')k= \begin{vmatrix} i& j&k \\ a & b & c \\ a' & b' & c' \end{vmatrix} $$

\edfn
\rm Операция $(u,v) \to [u,v]$ является билинейной (проверяется по определению) и антисимметричной (в каждой скобке в определении векторного произведения меняется знак), получаем $[u,u]=0$ и $[u,v]=-[v,u]$.
\erm

\dfn[Чисто мнимый кватернион]
Кватернион называется чисто мнимым, если его вещественная часть равна нулю. Часто обычный кватернион будем обозначать в виде $a + u$, где $u$ --- чисто мнимый.
\edfn

Рассмотрим произведение двух чисто мнимых кватернионов $u = (ai + bj + ck) \text{ и } v = (a'i + b'j + c'k)$\\
$uv = -(aa' + bb' + cc') + (bc' - cb') i + (ca' - ac')j + (ab' - ba')k =
-\lan u,v\ran+[u,v]$.

\dfn[Сопряжённый кватернион] Пусть $x$ --- кватернион, $x= a+bi+cj+dk = a + u$. Сопряжённым кватернионом называется $\ovl{x}= a-bi-cj-dk= \Re x - \Im x =a-u$. 
\edfn

\rm
$x \ovl{x} = (a + u) (a - u) = a^2 - u^2 = a^2 + \lan u, u \ran - [u, u] = a^2 + (b^2 i + c^2 j + d^2 k) - 0 = \ovl{x} x$
\erm

\dfn[Норма кватерниона] Определим норму кватерниона как $$||x||=\sqrt{x\ovl{x}}=\sqrt{ a^2+b^2+c^2+d^2}=\sqrt{\ovl{x}x}.$$
\edfn 

\rm 
   Все необходимые свойства нормы проверяются напрямую
\erm 

Норма кватерниона, как и модуль комплексного числа всегда положительны для ненулевых элементов. Это позволяет заметить, что

\dfn[Обратный кватернион] Если $0\neq x \in \mb H$, то $x^{-1}=\frac{\ovl{x}}{||x||^2}$. 
\edfn
Поскольку $x x^{-1} = \frac{x \ovl{x}}{||x||^2} = 1$\\ 

Таким образом мы получили первый (и для нас единственный) пример некоммутативного кольца с делением. Такие кольца называются телами. Напоминаю, что алгебра для нас ассоциативна и с единицей.

\rm
Связь между матричным и стандартным определением кватернионов\\
Пусть $v = \pmat \alpha & \beta \\ -\ovl{\beta} & \ovl{\alpha} \epmat \text{,тогда } u = \pmat a + b i & c + d i \\ -c + d i & a - b i \epmat = a + b i + c j + d k$\\
$\det v = \alpha \ovl{\alpha} + \beta \ovl{\beta} = a ^ 2 + b ^ 2 + c ^ 2 + d^2$\\
Отсюда $||x||=\sqrt{\det x}$
\erm

\lm[Норма мультипликативна] $||xy||=||x|| \, ||y||$ и $||x^{-1}||=||x||^{-1}$.
\proof 
$||xy|| = \sqrt{\det{x}} \sqrt{\det{y}} = \sqrt{\det{x} \det{y}} = \sqrt{\det{xy}}$, т.к. определитель мультипликативен.\\
Отсюда, $||x|| \, ||x^{-1}|| = ||x x^{-1}|| = ||1|| = 1$\\
\endproof
\elm

Отойдём немного в сторону и посмотрим на первое внутриалгебраическое применение кватернионов. Довольно давно был поставлен вопрос, какие натуральные числа бывают суммой четырёх квадратов. Лагранжем было доказано, что любое натуральное число допускает такое представление. Доказательство можно условно разбить на два этапа: \\
1) Показать такое представление для любого простого $p$.\\
2) Свести случай произвольного числа к случаю простого.

Мы не будем касаться первого пункта (см. \href{https://en.wikipedia.org/wiki/Lagrange%27s_four-square_theorem}{Теорема Лагранжа}).
	Разберёмся со вторым пунктом. Сначала докажем лемму
	\lm[Сумма четырёх квадратов] В кольце целых чисел произведение $(a^2+b^2+c^2+d^2)(e^2+f^2+g^2+h^2)$ снова есть сумма четырёх квадратов (а на самом деле и в любом коммутативном кольце с 1).
	\proof 
	Пусть $n_1 = a^2 + b^2 + c^2 + d^2, n_2 = e^2 + f^2 + g^2 + h^2 \text{, заметим, что } n_1 = ||a + bi + cj + dk|| ^ 2, n_2 = ||e + fi + gj + hk|| ^ 2$. По мультипликативности нормы $n_1 n_2 = ||(a + bi + cj + dk)(e + fi + gj + hk)|| ^ 2$. Легко заметить, что внутри получается некоторый кватернион с целыми коэффициентами (покуда каждый коэффициент является суммой произведения целых). Значит выражение будет равно сумме квадратов коэффициентов этого кватерниона. 
	\elm
	
\lm[Сведение произвольного числа к простому]
  Если мы доказали теорему для простых чисел, и что если теорема верна для двух чисел, то верна и для их произведения, то поскольку любое целое число представимо в виде произведения простых, то теорема верна и для любого целого числа (например, по индукции).
\elm
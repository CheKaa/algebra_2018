\section{
 Лемма про разделение на сомножители, чьи неприводимые множители имеют одинаковую степень.
}

\textbf{Необработанная версия из конспекта Константина Михайловича}

Это позволяет свести задачу разложения произвольного многочена над конечным полем к разложению на множители многочлена без кратных множителей. Действительно $\frac{f}{\Nod(f,f')}$ без кратных множителей. В свою очередь $\Nod(f,f')$ состоит из множителей двух типов -- чьи степени кратны $p$ и не кратны $p$. Первые встречаются как сомножители в  $\frac{f}{\Nod(f,f')}$ и легко находятся после получения его разложения. Из оставшихся множителей можно извлечь корень степени $p$ и перейти к разложению многочлена заведомо меньшей степени. 

Другое соображение позволяет свести задачу к разложению многочленов, чьи неприводимые сомножители имеют одинаковую степень. 

Заметим, что неприводимый многочлен степени $k$, является делителем $x^{q^k}-x$, но не являются делителями $x^{q^l}-x$ ни для каких $l<k$. Пусть $f$ -- многочлен степени ровно $n$ над $\mb F_q$. Без ограничения общности будем считать, что многочлен $f$ без квадратов.

\thrm  Существует полиномиальная от размера $f$ (то есть от $n\log q$) процедура, которая разделяет $f$ на сомножители $f_1, \dots, f_k$, что $f_i$ состоит из неприводимых сомножителей степени ровно $i$.
\proof  Переберём все многочлены вида $g_l=x^{q^l}-x$, где $l<n$. Пусть на шаге $l$ у многочлена $f$ нет неприводимых множителей степени меньше $l$. Тогда $\Nod(g_l,f)$ состоит из всех множителей степени ровно $l$ (они ведь входят в $f$ с кратностью 1). Тогда $\frac{f}{\Nod(g_l,f)}$ состоит из неприводимых множителей степени больше $l$ 

Осталось пояснить, как посчитать $\Nod(f,g_l)$. Проблема в том, что размер $g_l$ не полиномиально зависит от размера $f$. Здесь на помощь приходит следующее соображение: пусть $r$ -- остаток от деления $g_l$ на $f$. Тогда $\Nod(g_l,f)=\Nod(r,f)$. Степень же $r$ меньше  степени $f$. Осталось понять, как быстро найти $r$. Для этого надо вычислить $x^{q^l} \mod f$. Но для этого надо всего лишь возвести $x$ в степень $q^l$ в кольце $\mb F_q[x]/f$, что делается за $l\log q$ операций умножения в этом кольце, то есть за $l\log q$ умножений многочленов степени меньше $n$ и вычислений остатков для многочленов степени не более $2n$.
\endproof 
\ethrm  


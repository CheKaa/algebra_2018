\section{
	Сильно регулярные графы. Граф Петерсона и его спектр. Двудольность и спектр.
}

%это место выдаёт warning!
	
$A^2+(\mu-\lambda)A + (\mu-k)E=\mu J$,
$A_{|U}^2 + (\mu-\lambda)A_{|U} + (\mu-k)E = 0$ для 
$U = <(1,\cdots,1)>^{\bot}$\\
След степени == количество циклов == сумма собственных чисел с учетом кратности. $\lambda$ для 
$(v,\:w)$, $-\lambda$ для $(v,\:-w)$\\
	
	
\defn
$G$ - сильно регулярный $(n,~k,~\lambda,~\mu)$, если это $k$-регулярный граф на $n$ вершинах, причем у любых двух смежных вершин $\lambda$ общих соседей, а у двух несмежных -- $\mu$ общих соседей.
\thrm Матрица сильно $k$-регулярного графа удовлетворяет соотношению $A^2+(\mu-\lambda)A + (\mu-k)E=\mu J$, где $J$ -- это матрица из одних единиц.
\ethrm
\proof Посмотрим на $A^2$. $A^2_{ij}$ -- число путей из $i$ в $j$ длины 2. Если $i$ и $j$ смежны, то $A^2_{ij} = \lambda$, если не смежны, то $\mu$, если $i = j$, то $k$. Вычтем $kE$, получим нули на диагонали, вычтем $\lambda A$, получим нули на позициях ребер, добавим $\mu A + \mu E$, чтобы заменить 0 на $\mu$, получим $\mu J$.
\endproof
\rm Если граф -- сильно регулярный (для замечания, в общем случае верно и для регулярного), то у него есть собственный вектор $(1,\cdots,1)$. Ограничим $A$ на $U = <(1,\cdots,1)>^{\bot}$.
Тогда $A_{|U}^2 + (\mu-\lambda)A_{|U} + (\mu-k)E = 0$. Поймем, почему справа 0. У оператора $J_n$ ранка 1 есть $n$ собственных чисел, одно из них -- $n$ с единственным собственным вектором $(1,\cdots,1)$, остальные нули. При переходе к ортогональноу дополнению останутся только собственные числа 0, оператор, у которого все собственные числа 0 - нулевой.\\ 
\textbf{ здесь можно (и стоило бы) написать доказательство попроще, но я сходу не придумал}
\erm
---------------------------------------------------------------------------------------------------------------------------------------
\crl Граф Петерсена сильно регулярен ($k = 3,~\lambda=0,~\mu=1$), его спектр - ${-2,-2,-2,-2,1,1,1,1,1,3}$.
\ecrl
\proof
Подставим в полученное уравнение для ортогонального дополнения собственное число $A_{|U}$(если $A$ удовлетворяет уравнению, то и его собственные числа тоже). \\
$x^2+(1-0)x+(1-3)=0,~x^2+x-2=0$\\
$x_1 = -2,~~x_2 = 1$\\
Получим, что собственные числа - $-2$ и $1$ с кратностями $n_1$ и $n_2$ соответственно. Тогда имеет место следующая система:
\begin{equation*}
    \begin{cases}
        n_1+n_2+1=10,\\
        -2n_1 + n_2 + 3 = \tr(A) = 0.
    \end{cases}
\end{equation*}
Откуда $n_1 = 4$, $n_2 = 5$ (3 -- наибольшее собственное число, 1 -- его кратность)
\endproof
---------------------------------------------------------------------------------------------------------------------------------------
\thrm
Граф двудолен $\Leftrightarrow$ его спектр симметричен.
\ethrm
\proof
"$\Rightarrow$"\\
Пусть граф двудолен, тогда в каком-то базисе его матрицу смежности можно разбить на 4 подматрицы $A_{1,1} = 0$, $A_{1,2} = V$, $A_{2,1} = W$, $A_{2,2} = 0$ (нули - переходы из доли в себя же, $V$ и $W$ - переходы в соседнюю долю). Тогда несложно заметить $W = V^{T}$ (это никак не используется, просто прикольный факт).
$$\pmat 0 & V\\
 W & 0\epmat \pmat v\\
 w \epmat = \pmat \lambda v \\ \lambda w \epmat = \pmat V w \\ W v\epmat,\; \pmat v \\ w \epmat - \text{собственный вектор для}\: \lambda$$
 Подставим вместо $\pmat v \\ w \epmat$ $\pmat v \\ -w \epmat$
 $$\pmat 0 & V\\
 W & 0\epmat \pmat v\\ -w \epmat = \pmat V (-w) \\ W v \epmat = \pmat - V w \\ W v\epmat = \pmat -\lambda v \\ -\lambda (-w) \epmat$$
 Т.е. мы получили, что $- \lambda$ также собственное число, т.е. спектр симметричен.\\
"$\Leftarrow$"\\
Граф двудолен $\Leftrightarrow$ нет нечетных циклов $\Leftrightarrow$  $\tr(A^k) = 0\:\forall k \equiv 1 (\mod 2)$\\
Но $\tr(A^k)$ -- сумма собственных чисел $A^k$, коими являются $\lambda_i^k$ для собственных чисел $\lambda_i$ $A$. Т.к. спектр симметричен, то $\lambda_i = -\lambda_j\:\Rightarrow\lambda_i^k=-\lambda_j^k$ для нечетного $k$, то есть след будет нулем (для каждого есть такой же с минусом).
\endproof

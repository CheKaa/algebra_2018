\section{
 Единственность тензорного произведения. Размерность тензорного произведения.
}

Шпаргалка: TODO

\thrm Тензорное произведение единственно с точностью до изоморфизма.
\proof

Допустим, есть два тензорных произведения $(V_1 \otimes \dots V_n)_1$ и $(V_1 \otimes \dots V_n)_2$ и их полилинейные отображения $i_1$ и $i_2$. Тогда по определению существуют единственные линейные отображения $\hat{i}_1$ и $\hat{i}_2$, что
$$\hat{i}_1 \colon (V_1 \otimes \dots V_n)_2 \to (V_1 \otimes \dots V_n)_1, \ \hat{i}_1 \circ i_2 = i_1$$
$$\hat{i}_2 \colon (V_1 \otimes \dots V_n)_1 \to (V_1 \otimes \dots V_n)_2, \ \hat{i}_2 \circ i_1 = i_2$$

Нужно проверить, что $\hat{i}_1$ и $\hat{i}_2$ взаимно обратны. Проверим, что $\hat{i}_1 \circ \hat{i}_2 \circ i_1 = i_1$ и $\hat{i}_2 \circ \hat{i}_1 \circ i_2 = i_2$. Это будет значить, что $\hat{i}_1\hat{i}_2 = \id$.
$$\hat{i}_1 \circ (\hat{i}_2 \circ i_1) = \hat{i}_1 \circ i_2 = i_1$$
$$\hat{i}_2 \circ (\hat{i}_1 \circ i_2) = \hat{i}_2 \circ i_1 = i_2$$
\endproof
\ethrm

\thrm Пусть $e_{i1},\dots,e_{ik}$ базис $V_i$. Тогда $e_{1j_1}\otimes \dots \otimes e_{nj_n}$ базис $V_1 \otimes \dots \otimes V_n$. В частности, 
$$\dim V_1 \otimes \dots \otimes V_n= \prod_{i=1}^n \dim V_i.$$ 
\proof Прежде всего заметим, что набор $e_{1j_1}\otimes \dots \otimes e_{nj_n}$ является порождающей системой для тензорного произведения.
$v_1 \otimes \dots \otimes v_n = (\sum\limits_{j}c_je_{1j})\otimes v_2 \otimes \dots \otimes v_n =
\sum\limits_{j}c_je_{1j}\otimes v_2 \otimes \dots \otimes v_n$. И вот так же можно раскрыть все по полилинейности, чтобы остались только $e_{1j_1}\otimes \dots \otimes e_{nj_n}$.

Далее, по определению тензорного произведения,
$$\Hom(V_1,\dots,V_n, K) \cong \Hom(V_1\otimes \dots \otimes V_n,K).$$
Размерность последнего пространства совпадает с размерностью $V_1\otimes \dots \otimes V_n$ (это двойственное пространство). С другой стороны, полилинейное отображение $h \in \Hom(V_1,\dots,V_n, K)$ однозначно задаётся $\prod_{i=1}^n \dim V_i$  параметрами $h(e_{1j_1}, \dots,e_{nj_n})$. Комбинируя эти два факта получаем, что размерность $\dim V_1 \otimes \dots \otimes V_n$ есть $\prod_{i=1}^n \dim V_i$. Отсюда, любая порождающая система такого размера есть базис. В частности, набор $e_{1j_1}\otimes \dots \otimes e_{nj_n}$.
\endproof
\ethrm
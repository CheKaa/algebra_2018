\section{
 Степень расширения. Теорема о башне полей.
}

\dfn[Степень расширения] Пусть $L$ расширение поля $K$ (то есть $K$ подполе в $L$). $L$ можно рассматривать как векторное пространство над полем $K$ 
(есть сложение элементов и умножение на элементы $\in K$). 
Тогда размерность данного векторного пространства := $\dim_K L$ называется степенью $L$ над $K$ и обозначается как $[L: K]$. Если $[L: K]$ конечно, то говорят, что $L$ -- конечное расширение поля $K$.
\edfn

\thrm[О башне полей] Пусть дана башня расширений $K\leq L \leq M$. Тогда 
$$[M: K]=[M: L][L: K].$$
В частности, если $M$ конечно над $L$, а $L$ конечно над $K$, то $M$ конечно над $K$.
\ethrm

\proof
Возьмем $u_i$ -- базис $M$ над $L$, $v_j$ -- базис $L$ над $K$.
Докажем, что базис $M$ над $K$ -- все возможные произведения $u_i$ на $v_j$.

Докажем, что $u_i v_j$ -- порождающая система.
Возьмем $x \in M$. Так как $u_i$ -- базис $M$ над $L$, то $x = \sum_{i} \lambda_i u_i$, $\lambda_i \in L$. 
$v_j$ -- базис $L$ над $K$, поэтому $\lambda_i = \sum_j \alpha_j v_j$, $\alpha \in K$. Итого получили, что 
$x = \sum_{i} (\sum_j \alpha_{i,j} v_j) u_i = \sum_{i,j} \alpha_{i,j} u_i v_j$. 

Докажем, что $u_i v_j$ линейно независимы.
Пусть $\sum_{i,j} \alpha_{i,j} u_i v_j = 0$ и $\alpha_{i,j}$ не все равны 0. Тогда 
$\sum_i (\sum_j \alpha_{i,j} v_j) u_i = 0$. Так как $u_i$ линейно независимы, то все коэффициенты-суммы $\sum \alpha_{i,j} v_j$ равны нулю.
Но так как не все $\alpha_{i,j}$ равны нулю, то в какой-то из таких сумм есть ненулевые коэффициенты. Но
в таком случае такая сумма не может равняться нулю, так как $v_j$ -- базис и, следовательно, линейно независимы.

\endproof

\crl Пусть $[L: K]$ -- конечное расширение степени $n$, а $[M:K]$ -- степени $d$. Тогда, если $d \ndi n$, то $M$ не может быть подрасширением $L/K$.
\ecrl


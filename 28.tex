\section{
 Факториальность кольца многочленов над факториальным кольцом.
}

\textbf{Необработанная версия из конспекта Константина Михайловича}


\thrm Пусть $R$ -- факториальное кольцо. Тогда кольцо $R[x]$ факториально. Более того, имеет место следующее описание простых элементов кольца $R[x]$:\\
1)  $\cnt(f)=1$ и $f$ неприводим в $Q(R)[x]$.\\
2) $f=p \in R$ -- простой в $R$.
\proof 
Для начала покажем, что все указанные ситуации приводят к простым элементам в кольце $R[x]$ и что других простых и, более того, неприводимых не бывает.
Итак, пусть $f \in R[x]$ неприводим в $Q(R)[x]$. ($\Rightarrow$ $f$ прост в $Q(R)[x]$). Если $gh\di f$, то это же верно над $Q(R)$ и, можно считать например, что $g\di f$ в $Q(R)[x]$. Тогда $g= fk$. Теперь можно домножить на подходящую константу $g= (cf) (c^{-1}k)$ чтобы получить равенство в $R[x]$. Заметим, что $c(c^{-1}k)$ из $R[x]$, что показывает, что $g \di f$ в $R[x]$. Второй случай полностью следует из леммы Гаусса.


Теперь покажем, что любой элемент раскладывается в произведение указанных простых. Для этого сначала разложим $f$ в $Q(R)[x]$ в произведение неприводимых $f=\prod g_i$, $g_i \in Q(R)[x]$. Далее сделаем из $g_i$ элементы $\hat{g}_i$ из $R[x]$ с $cont(g_i)=1$, что $f=a\prod \hat{g}_i$. Заметим, что $a=cont(f)$ и, следовательно, лежит в $R$. Итого $f=a\prod \hat{g}_i$, где $ 0 \neq a \in R$. Осталось разложить $a$ (раскладывается в произведение простых второго типа).

Осталось показать единственность. Это следует лишь из того, что у нас есть разложение на простые. Действительно, если $f=\prod p_i=\prod q_i$, то $p_i \big| \prod q_i$ и благодаря простоте делит скажем $q_i$. Но тогда $p_ih=q_i$, откуда, благодаря неприводимости $q_i$ получаем, что $h$ обратим, то есть, что $p_i \sim q_i$. Тогда можно сократить на $p_i$ и продолжить по индукции. Отсутствие простых отличного от указанных типов следует теперь из единственности разложения.
\endproof
\ethrm


29) Редукционный признак неприводимости. Примеры. Признак Эйзенштейна.\\
1. $a_n \ndi p$, f - неприводим в $R/p[x]$ $\Rightarrow$ неприводим над $Q(R)$. $cont = 1$ и непр-ть над $Q(R)$ $\Rightarrow$ непр-ть над $R$ (см. степени $g$ и $h$). 2. $a_n \ndi p$, все $a_i \di p$ $i<n$, но $a_0\ndi p^2$, то многочлен $f(x)$ неприводим. Пусть $b_0 \ndi p$, см. min $c_s \ndi p$ и $a_s$.\\


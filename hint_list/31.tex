31) Лемма Гензеля. Разложение на множители при помощи леммы Гензеля.\\
    % Лемма Гензеля: Есть разложение $f \equiv_p gh,~(g, h) = 1 \Rightarrow$ можем найти $f \equiv_{p^k} \hat{g}\hat{h}$, где $\deg h= \deg \hat{h}$, $\deg g= \deg \hat{g}$, $\hat{g}\equiv_p g$, а $\hat{h}\equiv_p h$.\\
    Доказательство леммы: Индукция по k. Строим для $k + 1$. Помним, что $\forall f:~p^{k}f \equiv p^{k}\ovl{f} \pmod{p^{k + 1}}$.\\
    $\ovl{h} \equiv \hat{h}+p^ka(x) \Rightarrow \ovl{h}\ovl{g} \equiv \hat{g}\hat{h} + p^{k}(a(x)g + b(x)h)$. С другой стороны $f-\hat{g}\hat{h}=p^kc(x) \Rightarrow$ $a,~b$ берем из лп НОДа $g$ и $h$\\

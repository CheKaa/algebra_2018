19) Тензоры. Примеры. Координаты тензора. Замена переменной – случай тензора валентности (1,0).\\
$(p,0)$~--- полилин. форма, $(1, 1)$~--- лин. оп-р, $(2, 1)$~--- структ. алгебры. Переход: $x_{new} = Cx_{old}$. $e_i=\sum_{j=1}^nC_{ji}\hat{e}_j$, хотим $D:$ $e^i=\sum D_{ji}\hat{e}^j$. $e^k(e_i)=\delta_{ki}\Rightarrow \delta_{ki}=\sum_{j}C_{ji} \sum_{l}D_{lk}\hat{e}^l(\hat{e}_j)=\sum_{j,l}C_{ji}D_{lk}\cdot\delta_{lj}=\sum_{j}C_{ji}D_{jk}$ $E_n = C^TD\Rightarrow D = (C^{-1})^T$

07) Сингулярные значения и SVD-разложение.\\
$X^* = X^{\top}, \lan X^*e_i, e_j\ran = \lan e_i, Xe_j\ran, \sigma_i = \sqrt{d_i} > 0$ с.ч. $A^*A$. SVD $A \colon U \to V \ \exists$ о/н $u_i, v_j \colon$ матр $A = \Sigma(\sigma_{1..r}$ на диаг) ($X = L\Sigma R$). $e_i$ -- о/н c.в. $\lan Ae_i, Ae_j\ran = \lan A^*Ae_i, e_j\ran = \lan d_ie_i, e_j\ran, f_i = \frac{Ae_i}{\sqrt{d_i}}$ доп до базиса. $R = C^{-1} = C^{\top}, C$ столбцы $e_i$.\\

\section{
	Конечные поля... % не заменять на ldots
}

\textbf{Конечные поля. Число элементов. Основное уравнение. Эндоморфизм Фробениуса. Корни $x^{p^n} - x$ образуют подполе.}

Хорошо смотреть на мультипл. группу. Теорема Ферма для групп. Биномиальный коэф. делится на $p$ почти всегда.

\lm{Любое поле $K$ конечной характеристики $p$ содержит $\mb Z/p$ как подполе}

	\proof
	$\Im f : \mb Z \to K$, $f(x) = (x \mod p)$ -- искомое подполе. Фактически, кольцо порожденное сложением и умножением единичек. Легко проверить, что их сложение и умножение по модулю не выводит за пределы $\mb Z/p$. И $\Im f$ является полем, так как мы всегда умеет делить по модулю $p$.
	\endproof

\elm

\lm{Если $K$ -- конечное поле, характеристики $p$, то $|K| = p^n$, для некоторого $n$}

	\proof
	Рассмотрим порядки элементов $K$ как группы по сложению. Мы помним, что порядок элемента делит порядок группы. И обратно, для любого делителя порядка группы найдётся элемент такого порядка.

	Заметим, что $\forall a \neq 0 \in K,\ (p - 1)a \neq 0,\ pa = 0$. То есть нет элементов с порядком отличным от $p$, а значит у $|K|$ нет других делителей, кроме $p$. Тогда $|K| = p^n$.
	\endproof

\elm

\lm{Элементы $K = \mathbb{F}_{p^n}$ удовлетворяют уравнению $x^{p^n} - x = 0$}
	
	\proof
	В $K^*$ по теореме Ферма для групп любой элемент удовлетворяет $x^{p^{n} - 1} = 1$, т. к. $|K^*| = p^n - 1$. Это уже всё доказывает. Осталось домножить на $x$, чтобы единственный элемент не из $K^*$, а именно $0$ стал тоже удовлетворять уравнению.
	\endproof

\elm

\lm{Для кольца $L$ характеристики $p$ отображение $x \mapsto x^p$ -- эндоморфизм. А если $L$ поле, то автоморфизм. Называется эндоморфизм Фробениуса}

	\proof
	Произведение, очевидно, сохраняется: $(xy)^p = x^py^p$. Для суммы: $(x + y) ^ p = \sum\limits_k \binom{k}{p} x^ky^{p - k}$, но $p \mid \binom{k}{p}$, кроме $k \in \{0,\ p\}$. То есть $(x + y)^p = x^p + y^p$.

	Для полей это эндоморфизм, так как $x \mapsto x^{\frac{1}{p}} = x^{p^{n - 1}}$ -- обратный эндоморфизм. Ведь это многократный эндоморфизм Фробениуса $(x \mapsto x^p)^{\circ n - 1}$.
	\endproof

\elm

\lm{Если L -- поле характеристики $p$, то $K = \{ x \in L\ :\ x^{p^n} = x \}$ -- подполе.}
	
	\proof
	$0, 1 \in K$. Если $x = x^{p^n},\ y = y^{p^n}$, то $xy = (xy)^{p^n}$. Если $x = x^{p^n},\ y = y^{p^n}$, то $(x + y)^{p^n} = x^{p^n} + y^{p^n}$, так как возведение в $p^n$ -- эндоморфизм (перестановочен со сложением, в частности), а именно $n$-кратный Фробениус. Ну и обратный к $x$ -- $x^{p^n - 2}$, что видно из уравнения на элементы $K$.
	\endproof


\elm

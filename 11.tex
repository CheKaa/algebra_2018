\section{
 Неориентированные графы. Собственные числа связного графа. Два примера.
}

\textbf{Необработанная версия из конспекта Константина Михайловича}



\dfn Спектр графа -- это спектр его матрицы смежности $A(G)$.
\edfn


Для начала разберёмся с оценками и свойствами собственных чисел матрицы смежности. Здесь нам пригодится теорема Перрона. 


\lm Пусть граф $G$ связен. Тогда его максимальное собственное число положительно, не кратно и соответствующий собственный вектор имеет положительные координаты. Более того, все собственные числа графа по модулю меньше чем максимальная степень $d_{max}$. Граф $G$ регулярен тогда и только тогда, когда $d_{max}$ -- это его собственное число.
\elm
\proof Прежде всего отметим, что все собственные числа $G$ вещественные и максимальное собственное число положительно так как $\Tr A(G)=0$. Рассмотрим теперь матрицу $(A+\eps I)^{n-1}$. Это положительная матрица. Действительно в $(A+\eps I)^{n-1}_{ij}$ входит слагаемое $\eps^{n-1-l}$, где $l$ -- длина пути между $i$ и $j$. То же можно сказать и про большую степень $A+\eps I$. Максимальное с.ч. $A$ соответствует максимальному с.ч. $(A+\eps I)^l$ по крайней мере, если $\eps$ очень большое. Но тогда соответствующий собственный вектор $v$ положителен и максимальное собственное число $A+\eps I$ и, следовательно, $A$ не кратно. Далее $v$ положительный собственный вектор для всех $(A+\eps I)^{l}$, откуда получаем, что максимальное с.ч. у всех $(A+\eps I)^{l}$ наибольшее по модулю $(\lambda_1+\eps)^l>|\lambda_i+\eps|^l$. Переходя к пределу при $\eps \to 0$ получаем, что $\lambda_1^l \geq |\lambda_i|^l$. Осталось извлечь корень.

Почему же $\lambda_1 \leq d_{max}$? Пусть $x$ -- собственный вектор для числа $\lambda_1$. Тогда $Ax=\lambda_1 x$. Посмотрим, насколько мог измениться $x$ при домножении на $A$. Рассмотрим максимальную координату $x_i$. Имеем $\lambda_1 x_i= \sum a_{ij}x_j\leq d_{max} x_i$.

Предположим, что $d_{max}$ собственное число. Тогда в указанном выше неравенстве достигается равенство, то есть $x_i=x_j$ для соседних вершин. Но это значит, что вектор $(1,\dots,1)$ собственный, что бывает только в случае регулярного графа. 
\endproof


\exm \\
1) Спектр полного графа $K_n$ равен $n-1$ , $-1, \dots,-1$.\\
2) Спектр цикла длины $n$ равен $2\cos(\frac{2\pi l}{n})$. Действительно, матрица смежности цикла имеет вид 
$$C + C^{-1}, \text{ где } C=\pmat 0 & && 1 \\  1 & 0 && \\ & \ddots & \ddots & \\ & &1&0 \epmat \text{ матрица смежности ориентированного цикла}. $$
Так как собственные числа $\lambda_l$ для $C$ есть корни степени $n$ из единицы, то 
$$\lambda_l+\lambda_l^{-1}= \lambda_l+\ovl{\lambda_l}= 2\Re \lambda_l =2\cos\left(\frac{2\pi l}{n}\right)$$



\section{
 Внешняя и симметрическая степень... % не заменять на ldots
}

\textbf{Примеры. Лемма о проекторе для внешней степени. Формулировка для симметрической степени.}

\dfn Определим пространство $\Lambda^k V$ как подпространство $V^{\otimes k}$. Это подпространство выделяется следующими условиями -- для любой перестановки из $\sigma \in S_k$ и любого тензора $a\in \Lambda^k V$ верно, что $\sigma(a)=\sgn(\sigma)a$. Под $\sigma(a)$ подразумевается действие перестановки $\sigma$ на тензор $a$ перестановкой его компонент. Аналогично определяется подпространство $\Sym^k V \leq V^{\otimes^k}$, чьи элементы удовлетворяют свойству: $\sigma(a)=a$.
\edfn

\fct Полезно смотреть не на пространства $\Lambda^k (V)$ и $\Sym^k V$, а на пространства $\Lambda^k(V^*)$ и $\Sym^k(V^*)$, потому что они допускают привычную и наглядную интерпретацию --- их элементы это полилинейные функции со специальными свойствами.
\efct


\exm \\
1) Элемент $\Lambda^2(V^*)$ --- это просто кососимметрическая билинейная форма.\\
2) А элемент $\Sym^2 V^*$ -- это симметрическая билинейная форма или просто квадратичная форма.\\
3) Элемент $\Lambda^{\dim V} V^*$ -- это просто форма объёма на $V$.\\
4) Заметим, что продолжая аналогию с квадратичными формами, выбор базиса задаёт изоморфизм 
$$\Sym^k V^* \cong K[x_1,\dots, x_n]_{\deg =k}$$
с пространством однородных многочленов степени $k$ ($n$ -- размерность пространства). Последнее отображение устроено следующим образом -- элементу $a \in \Sym^k V^* $ сопоставим отображение, которое на векторе $v=x_1e_1+\dots+x_ne_n $ выдаёт $a(v,\dots,v)$. То есть 
$$a \to (v \to a(v,\dots,v)).$$
Это будет однородный многочлен от координатных функций $x_1, \dots, x_n$. Осталось заметить, что проекция тензора $e^{i_1}\otimes \dots \otimes e^{i_k}$ после применения такой операции --- это многочлен $x_{i_1}\dots x_{i_k}$.

\lm Имеет место проектор $Alt \colon V^{\otimes k} \to \Lambda^k V$ заданный формулой 
$$a \to \frac{1}{k!} \sum_{\sigma \in S_k} \sgn (\sigma) \sigma(a).$$
Аналогично отображение  
$$S\colon a \to \frac{1}{k!} \sum_{\sigma \in S_k} \sigma(a)$$
есть проектор на подпространство $\Sym^k V$.
\proof Докажем только первую часть. Прежде всего заметим, что $Alt$ принимает значение в подпространстве кососимметричных тензоров. Действительно 
$$\tau(Alt(a))=\frac{1}{k!}\sum_{\sigma \in S_k}\sgn(\sigma) \tau(\sigma(a))= \sgn{\tau} \frac{1}{k!}\sum_{\tau\sigma \in S_k} \sgn(\tau\sigma) \tau(\sigma(a))=\sgn(\tau) Alt(a).$$
Далее, покажем, что для любого кососимметричного тензора $a$ верно, что $Alt(a)=a$. Это покажет, что $Alt$ -- есть проектор и в его образе лежат все элементы из $\Lambda^k(V)$, то есть ровно то, что осталось показать. Итак пусть $a\in \Lambda^k(V)$. Тогда 
$$Alt(a)=\frac{1}{k!} \sum_{\sigma \in S_k} \sgn (\sigma) \sigma a=\frac{1}{k!} \sum_{\sigma \in S_k} \sgn^2(\sigma) a=a.$$
\endproof
\elm

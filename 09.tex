\section{
 Положительные матрицы. Теорема Перрона.
}

Шпаргалка: Док-во Перрона: положительность ($ A|x| \ge |x| \Rightarrow A|x| < \frac{A^n}{(1+\varepsilon)^n}A|x| \to 0$ противореч.), единственность (сонапр. коорд. $v \Leftarrow \sum\limits_{j} A_{kj} |v_j| = |\sum\limits_{j} A_{kj} v_j| $) и некратность (Жорд. клетки; либо $\exists c, i: |x_1 - c x_2|_i = 0$, либо $ A x_2 = x_2 + x_1 $)

\defn Матрицу $ A $ назовем положительной, если все ее элементы строго положительны.
\defn Матрицу $ A $ назовем неотрицательной, если все ее элементы неотрицательны.

\thrm[Перрона]
  Пусть матрица $ A $ положительна, $ \lambda $ ее наибольшее по модулю (!) собственное число. Тогда верны следующие утверждения:
  \begin{enumerate}
    \item Собственный вектор, соответствующий $ \lambda $, положителен.
    \item $ \lambda > 0,\ \lambda \in \mathbb{R} $.
    \item Модуль любого другого собственного числа строго меньше $ |\lambda| $.  
    \item $ \lambda $ -- не кратный корень характеристического многочлена $ A $.
  \end{enumerate}
\ethrm
\proof
  Пусть $ |\lambda| = 1 $ (иначе разделим все элементы матрицы на $ |\lambda| $, это никак не повлияет на доказываемые утверждения). Обозначим за $ x $ собственный вектор с собственным числом $ \lambda $.
  \begin{enumerate}
  	\item Покажем, что $ A|x| = |x| $ ($ |x| $ -- взять все координаты $ x $ по модулю). Это докажет первое утверждение. Для начала заметим, что 
  	$$ |x| = |Ax| \le A|x| $$ 
  	Левое равенство следует из определения $ x $, а правое неравенство нетрудно доказать, заметив, что при подсчете $ A|x| $ мы работаем только с положительными числами. Однако вместо неравенства мы хотим получить равенство. Пусть все же $ |x| < A|x| $. Обозначим $ z = A|x| $. Тогда $ y = z - |x| > 0 $, а значит и $ Ay > 0 $. Выходит, что существует $ \varepsilon > 0: Ay > \varepsilon z $, следовательно $ Ay = Az - A|x| = Az - z > \varepsilon z $ или же $ \frac{A}{1+\varepsilon} z > z $. Из-за положительности $ z $, $ A $ и $ \varepsilon $, применяя оператор $ \frac{A}{1+\varepsilon} $ к обеим частям неравенства, получим
  	$$ \frac{A^n}{(1+\varepsilon)^n}z > \frac{A^{n-1}}{(1+\varepsilon)^{n-1}}z > \ldots > z $$
  	Но у оператора $ \frac{A}{1+\epsilon} $ собственные числа меньше еденицы, поэтому $ \lim\limits_{n\to\infty} \frac{A^n}{(1+\varepsilon)^n}z = 0 $ (см. ниже). Получили противоречие с тем, что $ z = A|x| > 0 $.

  	\textbf{Касательно факта про предел.} \label{res20} Это следует из следствия 20 из конспекта прошлого семестра:

  	Пусть $ A $ -- вещественная (или комплексная) матрица с собственным числом $\lambda_1=1$ кратности 1, а все остальные собственные числа $A$ по модулю строго меньше 1. Если вектор $v= \sum c_i e_i$, где $e_i$ жорданов базис, то $$ \lim_{n \to \infty}A^nv = c_1 e_1 $$

  	\item Показав, что $ A|x| = |x| $, мы попутно доказали $ \lambda > 0,\ \lambda \in \mathbb{R} $
  	\item Рассмотрим собственное число $ \mu $ ($ |\mu| = 1 $) с собственным вектором $ v $. Тогда $ A|v|=|v|=|Av| $. Поэтому все координаты $ |Av| $ ненулевые (у $ |v| $ есть хотя бы одна ненулевая координата $ \Rightarrow $ у $ A|v| $ все координаты ненулевые $ \Rightarrow $ у $ |Av| $ все координаты ненулевые). Распишем равенство $ A|v| = |Av| $ для координаты $ k $: 
  	$$ \sum\limits_{j} A_{kj} |v_j| = v_k = |\sum\limits_{j} A_{kj} v_j| $$
  	Посмотрим на это равенство, как на равенство норм векторов в $ \mathbb{R}^2 $ ($ v_j $ комплексные, поэтому можно представить, что это вектора над $ \mathbb{R}^2 $). Известно, что такое ненулевое равенство достигается только когда $ v_j $ сонаправлены. Значит любая их линейная комбинация (над $ \mathbb{R} $) сонаправлена им, а поскольку координаты вектора $ Av $ как раз и есть линейные комбинации $ v_j $, то выходит, что координаты $ Av $ и $ v $ отличаются друг от друга на вещественный множитель. $ Av = \mu v \Rightarrow \mu \in \mathbb{R} \xRightarrow{A > 0} \mu = 1 $.
  	\item Пусть еденица кратный корень. Тогда в Жордановой форме матрицы $ A $ либо найдутся хотя бы две Жордановы клетки с числом $ \lambda = 1 $ на диагонали, либо одна клетка размера хотя бы 2 (с $ \lambda $ на диагонали). Первому случаю соответствует наличие двух несонаправленных собственных векторов $ x_1 $ и $ x_2 $ с собственными числами, равными $\lambda$. Тогда подберём $ c $, так что $ v = x_1 - c x_2 $ имеет нулевую координату. Получаем противоречие, так как $|x_1 - c x_2|$ -- неотрицательный вектор с собственным числом 1, но при этом с нулевой координатой ($ |v| = |Av| = A|v|$ и у $ |v| $ есть хотя бы одна ненулевая координата $ \Rightarrow $ у $ A|v| $ все координаты ненулевые $ \Rightarrow $ у $ v $ все координаты ненулевые).

  	Во втором же случае $ x_2 $ присоединён к $ x_1 > 0 $, то есть $ A x_2 = x_2 + x_1 $. Тогда имеем $ A^n x_2 = x_2 + n x_1 $. Это значит, что какие-то коэффициенты $ A^n $ растут по крайней мере линейно по $ n $. Но тогда и коэффициенты $ A^n x_1 = x_1 $ тоже растут по крайней мере линейно, что очевидно не так.
  \end{enumerate}
\endproof
\section{
 Три Петерсона %не надо писать сюда мать формулы
}

\textbf{$K_{10}$ не покрывается тремя Петерсонами.}

$\sum\limits_{i=1}^3 A_i = B$. Все рег $\Rightarrow$ общий с.в. $(1,\dots,1)$ для $P$ с.ч. 3, для полного с.ч. $9$. Сузим. Для $A_1$ и $A_2$ подпр. порожд. с.в. с с.ч. 1 $\cap$. Распишем для $u$ из $\cap$. $Bu = -u$ (натянуто на с.в. с с.ч. $-1$). $\Rightarrow$ с.в. для $A_3$ с с.ч. $-3$. Такого с.ч. нет.


Пусть есть граф $K_{10}$ (полный граф на 10 вершинах). В нём 45 рёбер. А ещё у нас есть граф Петерсона. В нём 15 рёбер. Хочется покрыть $K_{10}$ тремя Петерсонами. Но это невозможно, потому что:

\thrm $K_{10}$ не представим в виде дизъюнктного объединения $P_1\sqcup P_2\sqcup P_3$, где $P_1, P_2, P_3$~--- графы Петерсона.
	\proof
	Посмотрим на равенство $K_{10} = P_1 \sqcup P_2\sqcup P_3$ в матричном виде.

	Это означает, что $A_1 + A_2 + A_3 = B$, где $A_1, A_2, A_3$~--- матрицы Петерсонов, $B$~--- матрица полного.

	Но мы знаем для них собственные числа. Для Петерсона это $\{3, 1, 1, 1, 1, 1, -2, -2, -2, -2\}$, для $B = J_n - E$ ($J_n$~--- матрица из одних единиц)~--- это $\{n - 1 = 9, -1\text{ кратности 9}\}$

	Более того, все графы в нашем равенстве регулярные $\Rightarrow^{\text{(по лемме из какого-то прошлого билета)}}\ v=(1,\ldots, 1)^T$~--- собственный вектор для каждого из них. Для Петерсонов у него собственное число $3$, а для полного $n - 1 = 9$. 

	Тогда перейдём на подпространство $U = \langle v\rangle^{\perp}$. Его размерность 9.

	Рассмотрим $A_1$ и $A_2$. В них есть подпространства $U_1, U_2$ размерности 5, порождённые собственными векторами с собственным числом $1$ $\Rightarrow$ эти пространства пересекаются $\Rightarrow$ $\exists\ u\in U_1 \cap U_2$.

	$A_1 + A_2 = B - A_3$.

	$A_1u + A_2u = Bu - A_3u$.

	Знаем, что $A_1u = A_2u = u$. Так же знаем, что $Bu = -u$, так как ортогональное дополнение натянуто на подпространство из векторов с собственным числом $-1$.

	$\Rightarrow\ A_3u = -u-u-u = -3u$. Получается, что $u$~--- собственный вектор $A_3$ с собственным числом $-3$. Но у $A_3$ нет такого собственного числа. Противоречие. 
\endproof
\ethrm 

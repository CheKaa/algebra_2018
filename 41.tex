\section{
  Подполя данного конечного поля. Описание автоморфизмов... %не надо писать сюда формулы и ldots
}

\textbf{Подполя данного конечного поля. Описание автоморфизмов $F_p^n$.}

\textbf{Необработанная версия из конспекта Константина Михайловича}


\lm Пусть $K$ --- поле, $p=\chr K$ --- простое число. Тогда в $K$ есть подполе изоморфное $\mb Z/p$. Если к тому же $K$ --- конечное, то число элементов $|K|=p^n$ для некоторого натурального $n$. 
\elm
\proof Рассмотрим гомоморфизм $f\colon \mb Z \to K$. Ядро этого отображения это $p\mb Z$. Тогда $\im f\cong \mb Z/p$. Пусть теперь $K$ конечно. Рассмотрим $K$ как группу по сложению. Заметим, что порядок любого ненулевого элемента равен $p$. Так как нет элементов другого порядка, то в группе $p^n$ элементов по теореме Коши.
\endproof
Обычно доказательство идёт через линейную алгебру.


\thrm Существует и единственно (с точностью до изоморфизма) поле из $p^n$ элементов. Такое поле будем обозначать $\mb F_{p^n}$.
\ethrm

Начнём с леммы:
\lm Пусть $K$ поле из $p^n$ элементов. Тогда все элементы $K$ удовлетворяют уравнению $x^{p^n}=x$.
\elm
\proof Группа $K^*$ состоит из $p^n-1$ элементов. Тогда все элементы из $K^*$ удовлетворяют уравнению $x^{p^n-1}-1=0$. Домножая на $x$ добавляем неприкаянный 0.
\endproof


\lm Пусть $L$ --- кольцо характеристики $p$. Тогда отображение $x\to x^{p}$ является эндоморфизмом $L$. Это отображение называется эндоморфизмом Фробениуса. Если $L$ - конечное поле, то эндоморфизм Фробениуса является автоморфизмом. 
\elm
\proof Очевидно произведение переходит в произведение. $(x+y)^p=\sum_{i+j=p}{{p}\choose{i}} x^iy^j= x^p+y^p$ т.к. все промежуточные биномиальные коэффициенты делятся на $p$. Пусть теперь $L$ -- поле. Тогда заметим, что отображение $\Frob$ инъективно, так как в поле не бывает нильпотентов и, следовательно, по принципу Дирихле, биективно.
\endproof

\lm Пусть $L$ --- поле характеристики $p$. Тогда множество элементов из $L$ удовлетворяющих уравнению $x^{p^n}=x$ образует подполе в $L$.
\elm
\proof Обозначим рассматриваемое множество за $K$. Тогда $0,1\in K$. Очевидно, что $K$ замкнуто относительно умножения. Замкнутость относительно сложения следует из того, что $x^{p^n}$ есть композиция $n$ раз эндоморфизма Фробениуса. Значит $K$ --- подкольцо. Обратный к $x\neq 0$ имеет вид $x^{p^n-2}$, что следует из уравнения.
\endproof


\proof[{\color{red!80!black} Доказательство теоремы. Существование}]
Рассмотрим поле $\mb F_p=\mb Z/p$ и $x^{p^n}-x$ --- многочлен над ним. Тогда есть поле $L$ в котором   $x^{p^n}-x$ раскладывается на линейные множители. Рассмотрим $K$ --- подполе в $L$ состоящее из элементов удовлетворяющих уравнению $x^{p^n}=x$. В $K$ ровно $p^n$ элементов т.к. многочлен $x^{p^n}-x$ не имеет кратных корней.

\proof[{\color{red!80!black} Доказательство теоремы. Единственность}]
Пусть есть два поля $K$ и $L$ из $p^n$ элементов. Рассмотрим их мультипликативные группы. Они циклические порядка $p^n-1$. Пусть группа $K^*$ порождена элементом $\xi$. Тогда любой элемент заведомо является многочленом от $\xi$. Пусть $f$ -- минимальный многочлен $\xi$. Значит $K\cong\mb F_p[x]/f(x)$. Многочлен $f$ неприводим. $\xi$ --- его корень. Многочлен $x^{p^n}-x$ делится на $f$, так как у них есть общий корень $\xi$. Тогда у $f$ есть корни в любом поле из $p^n$ элементов, в частности в $L$. Тогда у нас есть гомоморфизм  $K\cong\mb F_p[x]/f(x)\to L$ переводящий $\xi$ в какой-то корень $f$. Этот гомоморфизм инъективен и по принципу Дирихле является биекцией. 
\endproof

\rm В частности, мы увидели, что любое конечное поле $\mb F_{p^n}$ имеет вид $K\cong\mb F_p[x]/f(x)$. Степень $f$ равна $n$ исходя их подсчёта числа элементов.
\erm


\thrm Поле $\mb F_{p^n}$ подполе $\mb F_{p^m}$ тогда и только тогда, когда $m\di n$. Такое подполе единственно.
\ethrm 
\proof
Если $\mb F_{p^n}$ подполе $\mb F_{p^m}$, то сравнивая степени расширения получаем, что $m \di n$. Обратно, возьмём в $\mb F_{p^m}$ подполе $\{x \in \mb F_{p^m} \,|\, x^{p^n}-x=0\}$. Очевидно, что любое подполе из $p^n$ элементов там содержится. Это даёт единственность. Для того, чтобы доказать существование покажем, что в указанном подполе $p^m$ элементов. Для этого заметим, что многочлен $x^{p^m}-x \di x^{p^n}-x$, если $m \di n$. Первый многочлен раскладывается на линейные множители над $\mb F_p$, откуда аналогичное свойство выполнено для второго многочлена. То есть у многочлена $x^{p^n}-x$ есть все $p^n$ корней в $\mb F_{p^m}$. Что и требовалось. 
\endproof



Покажем, что аналогичные свойства верны для расширений поля $\mb F_q$, где $q=p^n$. Большая их часть, естественно, сводится к расширениям $\mb F_p$, однако, в некоторых вопросах возникают дополнительные сложности. Основная из них -- наличие нетривиальных автоморфизмов у таких полей над $\mb F_p$. Начнём с леммы.

\lm Все автоморфизмы $\mb F_q$ над $\mb F_p$ имеют вид $\Frob_p^{\circ i}$, где $0\leq i \leq n-1$. 
\proof Заметим, что поле $\mb F_q$ порождено одним элементом $\mb F_q =\mb F_p[\alpha]$. Минимальный многочлен $\alpha$ над $\mb F_p$ обозначим за $f$, его степень равна $n$. 

Теперь, гомоморфизмы $\mb F_p[\alpha]$ определяются образами элемента $\alpha$, которые обязаны быть корнями того же многочлена $f$. Но таких корней в $\mb F_p[\alpha]$ не более $n$. Тогда и автоморфизмов не более $n$! Осталось показать, что мы нашли все $n$ возможных. Для этого предположим, что для всех элементов из $\mb F_q$ выполнено, что $\Frob_p^{\circ l} - \Frob_p^{\circ k}=0$, для $k,l<n$. Но это уравнение степени меньше $p^n$. Ему не могут удовлетворять все элементы $\mb F_q$!
\endproof
\elm

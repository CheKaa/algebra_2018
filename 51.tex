\section{
 Алгоритм декодирования Питерсона-Горенштейна-Цирлера.
}

\noindent {\bf Пример:}\\
Пусть $q=2$. Возьмём $n=2^4-1=15$ (то есть $m=4$). Для того, чтобы построить поле из 16 элементов рассмотрим многочлен $x^4+x^3+1$. Он неприводим над $\mb F_2$. Его корень $\alpha$ -- первообразный корень степени $15$ из единицы. Возьмём $l_0=1$ и $d=5$. Тогда необходимо найти минимальные многочлены для элементов $\alpha,\alpha^2,\alpha^3,\alpha^4$. Заметим, что минимальные многочлены для $\alpha,\alpha^2,\alpha^4$ одинаковы и равны $x^4+x^3+1$ так как два последних элемента есть образы предыдущих при автоморфизме Фробениуса. Минимальный многочлен для $\alpha^3$ равен $x^4+x^3+x^2+x+1$. 

В качестве многочлена задающего код БЧХ c конструктивным расстоянием $5$ можно взять $$g(x)=(x^4+x^3+1)(x^4+x^3+x^2+x+1).$$

\bupr Постройте проверочную матрицу для этого кода.
\eupr

Опишем простейший алгоритм декодирования -- алгоритм Питерсона-Горенштейна-Цирлера.

Пусть на вход мы получили многочлен $v(x)=u(x)+e(x)$, где $u(x)$ -- это пересылаемое сообщение, а $e(x)$ -- ошибка. Предположим, что $e(x)=e_{i_1}x^{i_1}+\dots +e_{i_t}x^{i_t}$ состоит из не более чем $t$ мономов, где $t<\frac{d}{2}$, то есть мы можем раскодировать сообщение. В этом случае $2t<d$. Для того чтобы узнать, что ошибки есть, мы вычисляем $v(\beta^{i})$, где $i\in \ovl{l_0, l_0+d-2}$. Но так как $u(x) \di g(x)$, то
$$v(\beta^{i})=u(\beta^{i})+e(\beta^{i})=e(\beta^{i}).$$
Итого мы знаем значения $e(\beta^{i})$. Обозначим за $$S_k= e(\beta^{l_0+k-1}), \,\, X_k=\beta^{l_0+k-1} \text{ и } Y_k=e_{i_k}.$$
В этих обозначениях $S_k$ перепишется как 
$$S_k=\sum_{i=1}^t Y_i X_i^{l_0+k-1}.$$

Если известны $X_k$, то из указанных выше уравнений легко найти $y_k$. Определитель матрицы этой системы есть, как и в доказательстве теоремы, определитель Вандермонда.

Итого, необходимо найти элементы $X_k$. Уравнения на $X_k$ не линейны. Наша задача ввести новые величины, однозначно определяющие $X_k$, на которые уже можно написать линейные уравнения. Для этого рассмотрим многочлен 
$$\Lambda(x)=(1-xX_1)\dots(1-xX_t)= 1+\Lambda_1x+\dots+\Lambda_tx^t.$$
Корни этого уравнения -- это величины $X_1^{-1},\dots,X_t^{-1}$. Если мы найдём $\Lambda_i$, то сможем найти $X_i^{-1}$ и, следовательно, найти $X_i$. Напишем тождество
$$0=\Lambda(X_l^{-1})=1+\Lambda_1 X^{l}+\dots+\Lambda_t X_l^{-t}.$$
Домножим его на $Y_lX_l^{\nu+t}$ и просуммируем по $l$. Имеем
$$ 0=\sum_{l=1}^t Y_lX_l^{\nu+t}+\sum_{l=1}^t\Lambda_1 Y_lX_l^{\nu+t+1}+\dots+\sum_{l=1}^t\Lambda_t Y_lX_l^{\nu}.$$
Теперь, коэффициенты при $\Lambda_i$ есть некоторые известные $S_k$. Если взять $\nu \in \ovl{l_0,l_0+t-1}$, то получим следующую систему уравнений
$$\left\{ \begin{array}{rcl}
-S_{t+1}&=& \Lambda_t S_{1}+\dots + \Lambda_1 S_{t}\\
&\vdots&\\
-S_{2t}&=& \Lambda_t S_{t}+\dots + \Lambda_1 S_{2t}\\
\end{array} \right.$$
Разрешимость этой системы зависит от её матрицы, которая имеет вид 
$$ \Sigma =\pmat
S_1 & \dots & S_t\\
\vdots & & \vdots\\
S_t & \dots & S_{2t}
\epmat.
$$
Осталось заметить, что 
$$ \pmat
S_1 & \dots & S_t\\
\vdots & & \vdots\\
S_t & \dots & S_{2t}
\epmat =  \pmat
1 & \dots & 1\\
\vdots & & \vdots\\
X_1^{t-1} & \dots & X_t^{t-1}
\epmat
\pmat
Y_1X_1 & \dots & 0\\
\vdots & \ddots& \vdots\\
0 & \dots & Y_tX_t
\epmat
\pmat
1& \dots & X_1^{t-1}\\
\vdots & & \vdots\\
1 & \dots & X_t^{t-1}
\epmat .
$$
Теперь, если ошибок было меньше чем $t$, то определитель этой матрицы 0 так как ранг средней матрицы меньше $n$. Если же ошибок ровно $t$, то определитель матрицы не ноль.

Итого мы получаем следующий алгоритм
\enm 
\item Запускаем цикл по $t$ от 1 до $\frac{d}{2}$. Вычисляем $S_1,\dots,S_{2t}$ и смотрим на определитель матрицы $\Sigma$. Если он ноль, то переходим к $t+1$. Если он не ноль, то решаем систему и находим $\Lambda_i$.
\item По $\Lambda_i$ находим $X_l^{-1}$ после чего находим позиции $i_l$. Это можно сделать подставив все возможные $\beta^i$ в $\Lambda(x)$. Если получился корень, то надо взять $-i \mod n$.  
\item Далее, решив систему линейных уравнений можно найти $e_{i_l}$. Осталось вычесть и найти $u(x)=v(x)-e(x)$.
\eenm


\section{
  Две оценки на размер максимального независимого множества.
}
    
    1. Натянуть подпространство на  множество, следствие из Куранта-Фишера, нулевая квадратичная форма
    2. Характеристический вектор множества, разложить по ортонорм. базису регулярного(!) графа с $u_1 = (1,\cdots,1) \frac{1}{\sqrt n}$
    
    
    \thrm Пусть $G$ - неориентированный граф на $n$ вершинах, а $A$ - симметричная матрица $n \times n$, такая, что на месте ребер в $A$ стоят нули (если $i$ соединено с $j$, то $A_{ij} = 0$), то размер независимого множества не превышает $\min(n - n_{+}, n-n_{-})$, где $n_{-}$ - количество отрицательных собственных чисел $A$, а $n_{+}$ - положительных.
    \ethrm
    \proof
    Рассмотрим подпространство, натянутое на вершины независимого множества (вектор, соответствующий вершине - это нули на тех позициях, с которыми вершина не соединена ребром, и единицы на тех, с которыми соединена). Тогда квадратичная форма $x^{\top}Ax$ при ограничении на это подпространство -- нулевая. Это дает нам, что все собственные числа ограничения - нули. Если независимое множество имеет размер $\alpha$, то и размер пространства, им порожденного - $\alpha$ (для каждой позиции не больше одного вектора с единицей в этой позиции, тогда все линейно независимы). Получим $\alpha$ собственных чисел $\mu_1, \cdots, \mu_{\alpha}$. Но по следствию из теоремы Куранта-Фишера $\mu_1 \le \lambda_1,\: \mu_2 \le \lambda_2, \cdots,\mu_{\alpha} \le \lambda_{\alpha}$, при этом все $\mu_i = 0$, т.е. количество неотрицательных собственных чисел (а это как раз $n - n_{-}$) не меньше $\alpha$. Аналогично можно доказать для неположительных (например, рассмотрев матрицу $-A$).
    \endproof
    \thrm Пусть $G$ - $k$-регулярный граф. Тогда размер его максимального независимого множества удовлетворяет неравенству
    $$\alpha(G) \le - \frac{\lambda_n n}{k - \lambda_n}$$
    \ethrm
    \proof
        Рассмотрим характеристичекий вектор $v$ максимального независимого множества $U$ ($v_i = 1$, если $i \in U$, 0 иначе).\\
        Понятно, что $v^{\top}Av = 0$, $v^{\top}v = \alpha$. Т.к. граф регулярный, то есть нормированный собственный вектор $u_1 = \frac{1}{\sqrt{n}}(1,\cdots,1)$. Тогда получим $\lan v, u_1 \ran = \frac{\alpha}{\sqrt n}$.\\
        Разложим $v$ по ортонормированной системе собственных векторов $u_1, \cdots, u_n$ $v = c_1 u_1 + \cdots + c_n u_n$. Тогда
        $$0 = v^{\top}Av = \sum c_i^2 \lambda_i = c_1^2 \lambda_1 + \sum\limits_{i=2}^{n} c_i^2 \lambda_i \ge c_1^2 \lambda_1 + \lambda_n \sum\limits_{i=2}^{n} c_i^2 = \lambda_1 \frac{\alpha^2}{n} +\lambda_n \sum\limits_{i=2}^n c_i^2$$
        Известно $\sum c_i^2 = \alpha$ (как $\lan v,v \ran$), тогда $\sum\limits_{i=2}^{n} c_i^2 = \alpha - \frac{\alpha^2}{n}$.\\ Получим $$0 \ge \lambda_1 \frac{\alpha^2}{n} + \lambda_n\left(\alpha - \frac{\alpha^2}{n}\right)$$
         $$(\lambda_1 - \lambda_n)\frac{\alpha}{n} \le -\lambda_n$$
         $$\alpha \le \frac{-\lambda_n n}{k - \lambda_n}$$
        
    \endproof

\section{
 Базис внешней степени. Формулировка для симметрической степени.
}

\dfn Пусть $e_1,\dots, e_k$ набор элементов из $V$. Определим элементы $e_1\wedge \dots \wedge e_k \in \Lambda^k V$ как образы при проекции $e_1\otimes \dots \otimes e_k$.
\edfn

\thrm Пусть $e_1,\dots, e_n$ базис пространства $V$. Тогда элементы $e_{i_1}\wedge \dots \wedge e_{i_k}$, где $i_1<i_2< \dots < i_k$ образуют базис пространства $\Lambda^k V$. В частности размерность $\dim \Lambda^k V = C^k_n$.
\proof Прежде всего заметим, что это действительно порождающая система. Для этого вспомним, что $e_{i_1,\dots,i_k}$ по всем возможным наборам $i_1,\dots,i_k$ порождают тензорное произведение $V^{\otimes k}$. Но раз они порождают тензорное произведение, то они порождают и его образ при проекции $Alt$ на пространство кососимметричных тензоров. Далее заметим, что $$Alt(e_{i_1,\dots i_k})= \sgn(\sigma) Alt(e_{i_{\sigma(1)},\dots i_{\sigma(k)}}).$$
В частности, если в наборе есть два повторяющихся индекса, то элемент проектируется в 0. Далее, это же соотношение даёт, что, возможно с точностью до знака проекция набора совпадает с проекцией упорядоченного набора. Таким образом, из набора образующих можно исключить неупорядоченные наборы и наборы с повторениями, что и требовалось.

Покажем линейную независимость. Для удобства введём обозначение: если $\Gamma\subseteq \{1,\dots,n\}$ размера, то за $e_{\Gamma}$ обозначим 
$$e_{\Gamma}= e_{i_1}\wedge \dots \wedge e_{i_k}, \text{ где } i_1<\dots<i_k \text{ и } \Gamma=\{i_1,\dots,i_k\}$$
Таким образом, каждому подмножеству мы сопоставили элемент $\Lambda^k(V)$. 

Пусть теперь 
$$\sum_{\Gamma} \alpha_{\Gamma} e_{\Gamma}=0.$$
Тогда расписывая эту сумму через базисные элементы $V^{\otimes k}$ имеем
$$\frac{1}{k!}\sum_{\Gamma} \alpha_{\Gamma} \sum_{\sigma \in S_{\Gamma}} \sgn(\sigma) e_{\sigma(i_1),\dots,\sigma(i_k)} =0 $$
Заметим, что все слагаемые соответствуют разным базисным элементам. Тогда, все коэффициенты равны нулю. В частности, коэффицинеты при $ e_{i_1, \dots, i_k}$, которые равны $\frac{1}{k!}\alpha_{\Gamma}$. 

\endproof
\ethrm

\thrm Аналогично, пусть $e_1,\dots, e_n$ базис пространства $V$. Тогда элементы образы тензоров $e_{i_1}\otimes \dots \otimes e_{i_k}$, где $i_1\leq \dots \leq i_k$ образуют базис пространства $\Sym^k V$.
\ethrm
